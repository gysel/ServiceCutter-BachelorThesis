\section{Project Status}

Every sprint one or two status meetings where held to track the progress.

\subsection{Kick Off \formatdate{15}{9}{2015}}

\textbf{Time:} 14:00 – 15:15

\textbf{Next Meeting:} \formatdate{21}{9}{2015} 10:30

\textbf{Attendees:} Olaf Zimmermann (ZIO), Lukas Kölbener (KOE), Michael Gysel (GYS)

\textbf{Agenda}
\begin{itemize}
\item Meetings, Sprints
\item Aufgabenstellung
\item Discussion on the subject of the thesis
\end{itemize}

\textbf{Notes}
\begin{itemize}
\item The automatic REST interface of Apache ISIS could be an inspiration
\item Vaughn Vernon provides his DDD samples on \href{https://github.com/VaughnVernon/IDDD_Samples}{GitHub}
\item The provided article by Subbu Allamaraju is a must-read.
\item General approach: Start with business requirements (Use Cases, User Stories) and transform them into Components (e.g. UML). These Components are then modeled into services.
\item UML Components approach: Identification, Specification, Realization. (Our tool would probably support the step from spec to realization)
\item Possible quality attributes: Coupling, Cohesion, Security, Performance, Data volatility, Frequency of updates, Monitoring, Reconciliation, Consistency, Data invariants, Data Volumes
\item Structurizer by Simon Brown is a possible input format
\item C4: Context, Container, Components, Classes
\item Swagger is a possible output of the tool
\item Idea: Quality attributes should be specified on the level of fields instead of entities.
\item New item in the reading list: \url{https://msdn.microsoft.com/en-us/library/ms954587.aspx} 
\item NFR: Number of entities/fields to be supported
\end{itemize}

\textbf{Decisions}
\begin{itemize}
\item MoM and Meetings should be handled the same way as in the SA.
\item Project management should be < 10\% of the overall effort.
\item Model reconciliation is out of scope!
\end{itemize}

\textbf{Tasks}
\begin{itemize}
\item GYS: Send invite for the next meeting
\item GYS/KOE: Suggest dates for the intermediate presentation (end Oct, beginning of Nov)
\item ZIO: Send Aufgabenstellung via mail to GYS/KOE
\item ZIO: Bring UML Components book to the next meeting
\item ZIO: Book a room for the next meeting (21.9. 16:30)
\end{itemize}

\subsection{Status Meeting \formatdate{21}{9}{2015}}

\textbf{Time:} 10:30 – 12:00
 
\textbf{Next Meeting:} \formatdate{19}{10}{2015} 14:00-16:00 (Zühlke Office)
 
\textbf{Attendees:} Olaf Zimmermann (ZIO), Lukas Kölbener (KOE), Michael Gysel (GYS), Wolfgang Giersche
 
\textbf{Agenda}
\begin{itemize}
\item Scope and content of the thesis 
\item Project Organization / Review Meetings
\item Intellectual property rights
\end{itemize}

\textbf{Notes}
\begin{itemize}
\item Coupling Criterion / architectural significant requirements
\subitem Data confidentiality
\subitem SPI, Sensitive Personal Information
\subitem Structural volatility
\item Critique on automated scoring systems for IT architecture:
\subitem Pseudo accuracy, for example when mapping non functional requirements
\subitem Only one of many relevant criteria considered
\subitem => Make a sensitivity analysis to analyze the impact of weight-parameters
\item Define exactly what is meant with entity, bounded context and aggregates are within the system
\item Check Spark framework for algorithms
\item Check triple graph grammar for model transformations
\item Possible modes of system: Suggested bounded contexts are local (Modules, Components) or separated by network (Remote/ Microservices)
\item The data used in multiple bounded contexts should be part of the \enquote{published language} (DDD)
\item Dr. Gerald Reif will take the role of the expert
\end{itemize}

\textbf{Decisions}
\begin{itemize}
\item The target audience (Persona) are architects which use the system as assistance in taking architectural decisions
\item The produced results will be published on Github under the Apache 2.0 license
\item Meetings 
\subitem 19.10.2015, 14:00-16:00 Office Zühlke 
\subitem 22.10.2015, 09:00-10:00 HSR
\subitem 19.10.2015, 11:00-12:00 HSR
\end{itemize}

\textbf{Tasks}
\begin{itemize}
\item GYS/KOE: Review \enquote{Aufgabenstellung} and send back to ZIO
\item GYS/KOE: Send invitation for meetings
\item GYS/KOE: Authorize ZIO on Github repositories
\item Giersche: Find example project(s) with good complexity level and documentation (domain model, user stories, NFRs...). Clarify if the project can be published as an example within the thesis.
\end{itemize}

\subsection{Status Meeting \formatdate{15}{10}{2015}}

\textbf{Time:} 11:00 - 12:00

\textbf{Next Meetings:} \formatdate{19}{10}{2015} 14:00, Z\"uhlke; \formatdate{22}{10}{2015} 09:00, HSR
 
\textbf{Attendees:} Olaf Zimmermann (ZIO), Lukas Kölbener (KOE), Michael Gysel (GYS)

\textbf{Agenda}

\begin{itemize}
\item Thesis \& documentation progress
\item Outcome discussion with O. Augenstein // Complexity, alternative approaches
\item Date of intermediate presentation (Zwischenpr\"asentation)
\item Interest of industry contacts
\item Idea workshop Monday @ Z\"uhlke
\end{itemize}


\textbf{Notes}
\begin{itemize}
\item The thesis title needs to be finalized by mid November.
\item \enquote{Service} is a good candidate for the things produced by the service cutter. Conflicting definitions exist for \enquote{Service} of which some focus on business capability and others on providing a remote API. 
\item Possible example project candidates are Netstal (\url{http://wwww.netstal.com}) and the master thesis by Jonas Biedermann.
\item The clustering approach has multiple flaws which are hard to overcome. A new approach using theory of sets is inspired by the discussion with Prof. O. Augenstein and will be analyzed within this sprint. 
\item Good visualization of the service cutter input and good traceability within the tool might support the taking and documentation of architectural decisions.
\item Andreas Rinkel will take the role of the \enquote{Gegenleser} for this bachelor thesis.
\item ZIO provided further ideas and research material on the topic of coupling criteria and decomposition. 
\end{itemize}
 
\textbf{Tasks}
\begin{itemize}
\item ZIO: Send draft for legal rights agreement (15.10)
\item ALL: Sign legal rights agreement (19.10)
\item GYS/KOE: Return "UML Components" book to ZIO by next week (22.10)
\item ZIO: Find date for intermediate presentation during 9-11. November (21.10)
\end{itemize}


\subsection{Status Meeting \formatdate{22}{10}{2015}}
\label{sec:status22102015}

\textbf{Attendees}: Wolfgang Giersche, Olaf Zimmermann (ZIO), Lukas Kölbener (KOE), Michael Gysel (GYS)
 
\textbf{Next Meeting}: 19.11.2015, 11:00 @HSR / Skype

\textbf{Agenda}

\begin{itemize}
\item Admin: Sign paper regarding usage rights
\item Demo: Prototype, development environment
\item Coupling Criteria Catalog
\item Algorithms and coupling quantification
\item Sample project (not discussed)
\end{itemize}
 
\textbf{Notes}

\begin{itemize}
\item Maintainability is important. Coupling Criteria should be changeable with an appropriate effort.
\item Findings discussion on the algorithm / approach.
\subitem Calculating a score of every possible service cut is not possible.
\subitem New term: Candidate Service Cut.
\subitem Idea W.G.: Start with a heuristic approach like “Substantiv-Clustering”, a set of Candidate Service Cuts derived from a graph cluster or an analysis using only one Coupling Criteria.
\subitem Document needs to explain why we did not “just use a graph cluster”.
\item Three step approach is probably required
\subitem Determine a set of Candidate Service Cuts (perform once) (needs to be a smart solution – not a brute force approach! E.g. 100 possibilities.)
\subitem Assess all Candidates with a given set of weights to come up with a score.
\subitem Evaluate all Candidates with priorities by CC to find best choices.
\item Idea: Shake solution to improve it.
\end{itemize}
 
\textbf{Decisions}

\begin{itemize}
\item It is not an exact science – a good solution is enough.
\item No complete freeze of the CC catalog as of now.
\item Performance for 100 Entities, 2000 Fields:
\subitem Calculate Candidate Service Cuts (Steps 1\&2) – less than 10 minutes
\subitem Evaluate Candidates with parametrized weights (Step 3) – less than 5 seconds
\end{itemize}
 
\textbf{Action Items}

\begin{itemize}
\item GYS/KOE: Send documentation to ZIO for a review by the end of Oct.
\item GYS/KOE: Enhance Coupling Catalog with Decomposition Impact, Example (maybe from trading system), Measurement/Quantification, Type (Distance/Proximity/Constraint)
\end{itemize}

\subsection{HSR Status Meeting \formatdate{11}{11}{2015}}

\textbf{Attendees}: Olaf Zimmermann (ZIO), Lukas Kölbener (KOE), Michael Gysel (GYS)
 
\textbf{Next Meeting}: tbd

\textbf{Agenda}

\begin{itemize}
\item Status of thesis / development
\item Sprint Goals 3 \& 4
\item Sample Project
\item Personas
\item Final presentation
\item Discussion Thesis Review
\end{itemize}

\textbf{Notes}

\begin{itemize}
\item The cargo tracking sample application is currently being reimplemented. Maybe take a look at it to use it as test project.
\item Can we apply the service cutter to the service cutter domain model? Good for credibility to test one's own systems. 
\item Personas are very good. Add an Enterprise Architect.
\item Papers on coupling: (focus on object oriented code, not architecture!)
	\begin{itemize}
		\item \url{http://ieeexplore.ieee.org/stamp/stamp.jsp?tp=\&arnumber=1605186}
		\item \url{http://ieeexplore.ieee.org/stamp/stamp.jsp?tp=\&arnumber=4021375\&tag=1}
	\end{itemize}
\item Ideas for a renaming of \enquote{data field}:
	\begin{itemize}
		\item Micro entity
		\item Nano entity
		\item Candidate Entity
		\item Nano service
	\end{itemize}
\item Describe every concept with an example
\end{itemize}

\textbf{Additional feedback by ZIO}
\begin{itemize}
\item Explain that criteria catalog does not aim for completeness.
\item Introduce example earlier .
\item Visualize coupling types.
\item Apply Service Cutter concepts/tool to Service Cutter design (as an additional validation step in last sprint).
\item Generalize from use case/user story as input (of busienss-level operations) to conceptual components (e.g. CRC cards)?
\item Make sure that all tool output can be reproduced, or that tool output can be consumed as input (e.g. in follow on projects or subsequent but delayed iterations in a project), consider previously proposed service cuts as one special type of legacy system constraint?
\end{itemize}

\textbf{Decisions}

\begin{itemize}
\item Notation
	\begin{itemize}
	\item Web uppercase
	\item service lowercase
	\item coupling criteria lowercase
	\item Service Cutter uppercase
	\end{itemize}
\item Final term instead of \enquote{data field} should be finalized in the next meeting.
\end{itemize}
 
\textbf{Action Items}

\begin{itemize}
\item ZIO: Find date for Abschlusspräsentation. (14. – 22. Januar)
\item ZIO: Send mentioned papers/books on pages 8 and 11.
\item GYS/KOE: Prepare suggestions for meetings in December.
\end{itemize}



\subsection{Status Meeting \formatdate{20}{11}{2015}}

\textbf{Attendees}: Wolfgang Giersche, Olaf Zimmermann (ZIO), Lukas Kölbener (KOE), Michael Gysel (GYS)
 
\textbf{Next Meeting}: \formatdate{4}{12}{2015}, Zühlke, Schlieren

\textbf{Agenda}

\begin{itemize}
\item Evaluation of Algorithms
\item Overview of Coupling Criteria Types and Scoring
\item Demo \enquote{Trading System}
\item Scope of Sprint 4
\item Various items
\end{itemize}

\textbf{Decisions}
 
Priority for the next sprint are:
\begin{itemize}
\item DDD Sample
\item Polishing of UI, small enhancements. See \enquote{Notes ZIO}.
\end{itemize}

\textbf{Notes ZIO}
\begin{itemize}
\item Define/explain and possibly challenge/refine all new names/concepts, expsecially those exposed in UI, e.g. \enquote{Variant}, \enquote{Field Group}, \dots
\item If input is a set of nano entities (new name of data field, which is ok with me), what is the output? Services? Microservices? Entitites? Bounded contexts? Current UI says \enquote{Service A}, \enquote{Service B} etc. (which is ok, but input/output terminology should be consistent and symmetric) 
\item Feature requests/ideas incl. UI enhancements:

	\begin{itemize}
	\item Save output (service cuts suggested by alg.) in a machine-processable form, JSON at a minmum (so that next tool in chain can pick them up); ideally, this output takes the form of serialized CRC cards and/or Swagger definitions 
	\item Ability to store and reload all settings done by user, e.g. prioritizations
	\item Stepwise forward/backward button, or before/after feature to support incrementa design and what-if analyses (as a service modeler, I want to see how service cut changes when I update one setting at a time, and I also want to be able to undo my last changes)
	\item Show output of several algorithms in parallel/in comparison (workaround/simpleo solution: two browser tabs wordking with same model in backend)
	\item \enquote{Reset to default settings} button (e.g. for prioritizations)
	\item Graph metrics (analysis) with simple aggregation gree-yellow-red \enquote{how good is the current service cut} (according to some rules of thumb from OOAD or SOA design methods; I  can provide some input in next meeting)
	\item Make defaults and score values [-10…10] configurable (e.g. in config file or system property)
	\item Ability to define own variants (with weight scores), e.g. what if I want to specific a security classification \enquote{very critical} in addition to those shown in demo
	\end{itemize}

\end{itemize}

\textbf{General Notes}

\begin{itemize}
\item Coupling types visualization:

	\begin{itemize}
	\item Order of rows: Domain, Quality, Infrastructure, Security
	\item Empty boxes: Explain! (List is not exhaustive – but other types were not part of the outcome of the workshop)
	\item Cohesion instead of Cohesiveness? (Cohesion already has a meaning)
	\item Goal: only one term in the final version
	\end{itemize}

\item Scoring

	\begin{itemize}
		\item -10 to 10 is a good range for human beings. Technically -1 to 1 would be at least as good.
		\item Explain reasons why Fibonacci numbers were selected for the priorities. (Quote similar paper where Fibonacci were used for priorities or perform an A/B testing.)
		\item Document sample with a few priorities and a second example with all/high priorities used.
		\item Thoroughly document justify the scoring process.
		\item The weights are opinions. Make configurable and document. 
		\item Introduce all introduced terms such as variants, field groups, score, weight.
		\item Find better term than “variant”.
		\item Discussion: Data volumes, document why it should scale. Do tests.
		\item Document defaults and make them configurable.
	\end{itemize}

\item Document the import and how it works in detail. Add defaults to CC cards?
\item Document the trading example tests.
\item Are CRC card similar to our output?
\item \enquote{Parameters} instead of \enquote{Params}
\item Document the different Algorithms.
\item Document the value of the Service Cutter to an architect! Present some unexpected but good decompositions. Also educational aspects.
\item Document the value of the Service Cutter in the process of going from a monolith to a service oriented architecture. 
\item Add some kind of hint on whether the suggested cuts are good or not (traffic lights).
\item Source code: Should not be embarrassing but does not have to be perfect. It still is a prototype.
\item The tool should produce some kind of output.

\end{itemize}

\subsection{Status Meeting \formatdate{4}{12}{2015}}

\textbf{Attendees}: Wolfgang Giersche, Olaf Zimmermann (ZIO), Lukas Kölbener (KOE), Michael Gysel (GYS)

\textbf{Agenda} 

\begin{itemize}
\item Progress Update
\item DDD Sample “Cargo Tracking” Analysis 
\item Outlook: last week
\end{itemize}

\textbf{Decisions}

\begin{itemize}
\item It is sufficient that not all described coupling criteria are implemented within the Service Cutter. The main goal of the thesis is the proof of concept and not completeness.
\item Priorities:
	\begin{itemize}
	\item Wolfgang Giersche: Additional Leung layer and analysis Leung vs. Girvan-Newman.
	\item ZIO: Being able to work with Service Cutter, change settings, store results and use results for further processing in a toolkit chain. Being able to give the tool to another person to work with.
	\item Decision: Use the remaining time to wrap up Service Cutter functionality and usability with future work in mind. (Risk is too high and resources are too short.)
	\end{itemize}

\end{itemize}


\textbf{Action Items}

\begin{itemize}
\item ZIO: Send invitation for the final presentation to Wolfgang Giersche.
\item GYS/KOE: Send documentation and tool access for review to ZIO until 07.12.15 5pm.
\end{itemize}

\textbf{Notes }
\begin{itemize}
\item Document results of tests with expectation comparison. Define in a measurable way what a \enquote{good} or \enquote{bad} result is (test systems analysis).
\item Comparison Leung vs. Girvan-Newman: Different quality of output with exactly the same input. 
\item Document comparison with same number of services.
\item DDD Sample use cases have been reverse engineered from the sample source code.
\item Conduct a sensitivity analysis on priorities and weights.
\item Document evolutionary architecture, Fowler's \enquote{Monolith to Microservices}-approach with number of services as a parameter. 
\item Not defining number of services but receiving a suggestion by the Service Cutter has the advantage of avoiding to prejudice the solution.  
\item Document further analysis and integration of MCL in \enquote{Future Work} chapter.
\item Additional Layer for Leung in \enquote{Future Work} chapter $\rightarrow$ Machine Learning?
\item Document which criteria have been defined in the workshop and which were added later.  
\item Document \enquote{Communication} Criteria separately as \enquote{Candidate Criteria} to be implemented in \enquote{Future Work}.
\end{itemize}
