\section{Project Status}

Every sprint one or two status meetings where held to track the progress.

\subsection{Kick Off \formatdate{15}{9}{2015}}

\textbf{Time:} 14:00 – 15:15

\textbf{Next Meeting:} \formatdate{21}{9}{2015} 10:30

\textbf{Attendees:} Olaf Zimmermann (ZIO), Lukas Kölbener (KOE), Michael Gysel (GYS)

\textbf{Agenda}
\begin{itemize}
\item Meetings, Sprints
\item Aufgabenstellung
\item Discussion on the subject of the thesis
\end{itemize}

\textbf{Notes}
\begin{itemize}
\item The automatic REST interface of Apache ISIS could be an inspiration
\item Vaughn Vernon provides his DDD samples on \href{https://github.com/VaughnVernon/IDDD_Samples}{GitHub}
\item The provided article by Subbu Allamaraju is a must-read.
\item General approach: Start with business requirements (Use Cases, User Stories) and transform them into Components (e.g. UML). These Components are then modeled into services.
\item UML Components approach: Identification, Specification, Realization. (Our tool would probably support the step from spec to realization)
\item Possible quality attributes: Coupling, Cohesion, Security, Performance, Data volatility, Frequency of updates, Monitoring, Reconciliation, Consistency, Data invariants, Data Volumes
\item Structurizer by Simon Brown is a possible input format
\item C4: Context, Container, Components, Classes
\item Swagger is a possible output of the tool
\item Idea: Quality attributes should be specified on the level of fields instead of entities.
\item New item in the reading list: \url{https://msdn.microsoft.com/en-us/library/ms954587.aspx} 
\item NFR: Number of entities/fields to be supported
\end{itemize}

\textbf{Decisions}
\begin{itemize}
\item MoM and Meetings should be handled the same way as in the SA.
\item Project management should be < 10\% of the overall effort.
\item Model reconciliation is out of scope!
\end{itemize}

\textbf{Tasks}
\begin{itemize}
\item GYS: Send invite for the next meeting
\item GYS/KOE: Suggest dates for the intermediate presentation (end Oct, beginning of Nov)
\item ZIO: Send Aufgabenstellung via mail to GYS/KOE
\item ZIO: Bring UML Components book to the next meeting
\item ZIO: Book a room for the next meeting (21.9. 16:30)
\end{itemize}

\subsection{Status Meeting \formatdate{21}{9}{2015}}

\textbf{Time:} 10:30 – 12:00
 
\textbf{Next Meeting:} \formatdate{19}{10}{2015} 14:00-16:00 (Zühlke Office)
 
\textbf{Attendees:} Olaf Zimmermann (ZIO), Lukas Kölbener (KOE), Michael Gysel (GYS), Wolfgang Giersche
 
\textbf{Agenda}
\begin{itemize}
\item Scope and content of the thesis 
\item Project Organization / Review Meetings
\item Intellectual property rights
\end{itemize}

\textbf{Notes}
\begin{itemize}
\item Coupling Criterion / architectural significant requirements
\subitem Data confidentiality
\subitem SPI, Sensitive Personal Information
\subitem Structural volatility
\item Critique on automated scoring systems for IT architecture:
\subitem Pseudo accuracy, for example when mapping non functional requirements
\subitem Only one of many relevant criteria considered
\subitem => Make a sensitivity analysis to analyze the impact of weight-parameters
\item Define exactly what is meant with entity, bounded context and aggregates are within the system
\item Check Spark framework for algorithms
\item Check triple graph grammar for model transformations
\item Possible modes of system: Suggested bounded contexts are local (Modules, Components) or separated by network (Remote/ Microservices)
\item The data used in multiple bounded contexts should be part of the \enquote{published language} (DDD)
\item Dr. Gerald Reif will take the role of the expert
\end{itemize}

\textbf{Decisions}
\begin{itemize}
\item The target audience (Persona) are architects which use the system as assistance in taking architectural decisions
\item The produced results will be published on Github under the Apache 2.0 license
\item Meetings 
\subitem 19.10.2015, 14:00-16:00 Office Zühlke 
\subitem 22.10.2015, 09:00-10:00 HSR
\subitem 19.10.2015, 11:00-12:00 HSR
\end{itemize}

\textbf{Tasks}
\begin{itemize}
\item GYS/KOE: Review \enquote{Aufgabenstellung} and send back to ZIO
\item GYS/KOE: Send invitation for meetings
\item GYS/KOE: Authorize ZIO on Github repositories
\item Giersche: Find example project(s) with good complexity level and documentation (domain model, user stories, NFRs...). Clarify if the project can be published as an example within the thesis.
\end{itemize}
