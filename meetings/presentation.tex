\section{Presentation}

\subsection{Intermediary Presentation \formatdate{11}{11}{2015}}

\textbf{Attendees:} Gerald Reif (ipt), Andreas Rinkel (HSR), Olaf Zimmermann (HSR), Lukas Kölbener, Michael Gysel

\textbf{Agenda}

\begin{itemize}
\item Introduction
\item Thesis
\item Approach
\item Demo
\item Next steps
\end{itemize}


\textbf{Notes}

\begin{itemize}
\item Clearly introduce the goal of the thesis / the tool.
	\begin{itemize}
	\item We aim to support the architect --– not replace him.
	\item We aim to decompose services --– not compose them.
	\item Our goal is to build good services. Composing them into applications, defining technology or the communication between services is out of scope.
	\end{itemize}
\item Answer question: What is the tool going to do for me?
\item Phrase: \enquote{Modell- oder Eventgetrieben}. Where do we stand?
\item Properly introduce the term service in the thesis. Also relate to operations and business logic.
\item Define entity. (Not the db term! Relate to the DDD entity. Also includes logic – not just data.)
\item Answer question: Why do I need services at all?
\item Introduce the handling of operations early in documentation.
\item How are future iterations reflected? Can we feed the service cut output back into the system for the next iteration? (Feature Request!)
\item Document how the coupling criteria catalog was developed.
\item Document the comprehensiveness of the cc catalog.
\item Possible real world problem: Too much data is required. Where do we get it from, how to we enter it? Provide good defaults!
\item Create images/icons for types of coupling criteria.
\end{itemize}


\textbf{Presentation}

\begin{itemize}
\item Introduce target group (architect, developer, designer) earlier.
\item Flip software layers diagram vertically.
\item Spell out abbreviations.
\item Introduce cited persons.
\end{itemize}