\section{Risk Management}

To assess the risk associated with this project, a comprehensive list of possible risks and their mitigations is described in the following Section. 

The risk assessment shown in Table \ref{tab:toprisks} is based on a literature survey\cite{risk} taken in 2011. Two custom lists of project specific risks are documented in Table \ref{tab:projmgmtrisks} and Table \ref{tab:projtechnicalrisks}.

\begin{table}[H]
\begin{tabular}{|c|p{80pt} p{140pt} p{140pt}|}
\hline \# & Risk & Impact & Evaluation \& Mitigation \\ 
\hline 1 & Stakeholders have opposing requirements. & It is impossible to prioritise requirements. & Discuss implementation approaches and priorities in meetings with all stakeholders present. \\ 
2 & Wrong priorities are defined and unimportant features implemented first. & Important features are left out. & Validate priorities and functionality with all stakeholders on a regular basis. \\ 
3 & The domain area is more complex than expected. & The implementation takes more time than expected and not all required features can be implemented. & The iterative approach helps to focus to maximize business value at all times. \\ 
4 & A team member faces health issues and cannot continue to work on the project. & The project scope is impossible to fulfill. & The project scope has to be renegotiated with Z\"uhlke and HSR. \\ 
5 & Project infrastructure outage & JIRA, GitHub or the project server goes down and the progress is therefore delayed.  & All components are supported by a company and professional support should therefore be available quickly.  \\ 
6 & Personal infrastructure failure & A personal laptop of one of the team members stops working. & HSR desktops are available and can be used as replacement hardware. Furthermore an personal spare notebook owned by Michael Gysel is available as well.  \\ 
\hline
\end{tabular}
\caption{Project-Specific Management Risks}
\label{tab:projmgmtrisks}
\end{table}

\begin{table}[H]
\begin{tabular}{|c|p{80pt} p{140pt} p{140pt}|}
	\hline \# & Risk & Impact & Evaluation \& Mitigation \\ 
	\hline 1 & Calculating service cuts using a clustering algorithm does not produce usable results. & An important functional part of the service cutter cannot be implemented. & Also evaluate other types of algorithms and functional alternatives. Discuss ideas early on with experts to validate feasibility. \\ 
	2 & Unstable development environment & Analysing and fixing problems takes too much time and causes a delay in the project plan. & Make use of established development tools and agree on a common version of Java, the IDE and plugins. \\ 
	3 & Architectural decisions introduce unnecessary complexity. & Implementation of functionality is  & Assess every architectural decision regarding its impact towards the complexity. Where possible prefer established technologies. \\ 
	4 & Developed source code is not covered with unit tests. & Future refactorings are hard to perform without appropriate test coverage. Unit tests serve as a functional documentation as they specify the intended behavior of a piece of code. & All written software should be covered with automated unit or integration tests. The code coverage plugin of Jenkins helps to monitor the test coverage. \\ 
	\hline
\end{tabular}
\caption{Project-Specific Technical Risks}
\label{tab:projtechnicalrisks}
\end{table}

\begin{table}[H]
\begin{tabular}{|c|p{80pt} p{140pt} p{140pt}|}
\hline \# & Risk & Impact & Evaluation \& Mitigation \\ 
\hline 1 & Misunderstanding of requirements & The final product consists of features that do not comply with the requirements of the customer. & User Stories are reviewed by the customer prior to the implementation and review meetings are held at the end of every sprint. \\ 
2 & Lack of management commitment and support & The project lacks funding or resourcing. & The project sponsor and other stakeholders committed to attend status and review meetings whenever allowed by their schedules. \\ 
3 & Lack of adequate user involvement & Requirements and solutions cannot be validated with users. & workshop? \\ 
4 & Failure to gain user commitment & End users may not be able to provide required progress reviews or required contributions towards the requirement specification. & ? \\ 
5 & Failure to manage end user expectation & End users may not be able to use the product or refuse to do so. & The end users can influence the priorities of the user stories. This ensures that the most relevant features will be tackled first. \\ 
6 & Changes to requirements & Already implemented functionality turns out to be unnecessary or based on wrong assumptions. & Likely to happen as the stakeholders have varying priorities and conceptions of the product to be developed. Has to be mitigated by reviewing requirements as part of the regular meetings. Absent stakeholders need to be informed of all decisions taken at such meetings. \\ 
7 & Lack of an effective project management methodology & Project is delayed and generated business value is impacted. & The project will be managed using Scrum, a methodology that is already known to all involved parties. \\ 
\hline 
\end{tabular} 
\caption{Top Software Risks Evaluation}
\label{tab:toprisks}
\end{table}

\subsection{Risk Assessment}

%As all mitigation measurements described in the last Section were implemented and
%therefore none of the described risks put the project seriously at risk. The following risks
%impacted the project nevertheless:

todo

%TODO

% All taken measures and used methodologies proved to be good choices to ensure good
% project management. The project did not suffer from management overhead nor were
% important aspect left out or forgotten during the project