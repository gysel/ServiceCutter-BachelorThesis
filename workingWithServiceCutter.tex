\chapter{Working with the Service Cutter} 

Benefit: structured catalog, the right questions are asked.


%TODO Potential contents of this chapter:
\begin{itemize}
	\item How do you organize such an interview?
	\item Documentation of input format?
	\item When to choose which algorithm?
	\item Recommendations for working with the priorities.
\end{itemize}

This section outlines our proposed usage schemes.

%TODO: Does this chapter make sense? it could cover the process of working with the service cutter, finding the needed information, intepreting the results and using it for decision finding and documentations (ADs). Giersche highly recommended/requested this. 
\subsection{Data Population}

\begin{enumerate}
	\item UML Class Diagram
	\item Use Cases or User Stories
	\item Contextual information
\end{enumerate}

How do you initially populate all the coupling data? LOTS of knowledge is required!

Hard to keep it up to date.

\subsection{Usage Scenarios}

- from monolith to microservices
- green field services, replicable priorities

\subsection{Interpreting the Result}

?
\subsection{Development Iterations}

What to do with the output? How to feed it back into the loop in the next release cycle?


After introducing usage scenarios of the service cutter, the next chapter assesses the decomposition approach and the practicability of the Service Cutter