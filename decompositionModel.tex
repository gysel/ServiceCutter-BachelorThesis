\chapter{Decomposition Model}
\label{cha:decomposition}

The decomposition model describes the quality attributes of good service decomposition solutions and the criteria leading to such. This chapter starts with an overview over all defined coupling criteria and concludes with the definition of a good decomposition solution. 

A coupling criterion describes an architecturally significant requirement why two nanoentities should or should not be owned by the same service. These criteria define the semantic model on which the Service Cutter is built on. 

The coupling criteria are a product of literature research and a workshop assembling the collective software architect experience of our thesis advisor, our industry partner, and us. We transformed the resulting ideas into the following structured catalog.

\section{Catalog Overview}
\label{subsec:couplingCriteriaOverview}

We arranged the coupling criteria in a grid as shown in Figure \ref{fig:cc_grid}.

The grid columns represent the following partitions:

\begin{description}
	\item[Cohesiveness] - Criteria describing the cohesiveness of nanoentities and therefore why they should belong to the same service. 
	\item[Compatibility] - Criteria describing the divergent characteristics of nanoentities. The service should not contain nanoentities with different characteristics. Examples for incompatible characteristics are \textit{High}, \textit{Eventually}, and \textit{Weak} for the criterion \textit{Consistency Criticality}.
	\item[Constraints] - Criteria describing constraints which enforce that groups of nanoentities must be distributed amongst different services or form a service by itself.
	\item[Communication] - Criteria describing which nanoentities are suitable to be used as part of the published language shared between services. 
\end{description}

\clearpage
The rows are inspired by the \enquote{4+1 View Model of Software Architecture} by Kruchten\cite{fourPlusOne}. Domain is an enhancement of the \textit{Logical View}, quality resembles the \textit{Process View} and physical matches the \textit{Physical View} but also includes predefined service constraints. Security is included in the \textit{Development View} by Kruchten. We decided to promote it to a separate layer as it was a prominent requirement in our workshop and other aspects of the \textit{Development View} were not relevant for our application.

\begin{description}
	\item[Domain] Criteria describing nanoentities from a business domain perspective.
	\item[Quality] Criteria describing the quality requirements of a nanoentity directly or related to a use case. Non-functional requirements are predominantly represented in this row.
	\item[Physical] Criteria describing the physical or technological aspects of nanoentities.
	\item[Security] Criteria describing nanoentities from a security perspective.
\end{description}

\begin{figure}[H]
	\includegraphics[scale=0.5]{diagrams/CouplingCatalog.pdf}
	\caption{Coupling Criteria Catalog}
	\label{fig:cc_grid}
\end{figure}

\clearpage
\section{Coupling Criteria Cards}
\label{sec:couplingCriteriaCards}

We specified all coupling criteria listed in the catalog as \enquote{CC cards} like the following:

\newcommand{\ccCard}[8] {
\begin{minipage}{\linewidth}
	\begin{framed}
	\textbf{#1 #2}
	
		\begin{description}[leftmargin=!,labelwidth=\widthof{\bfseries User Representation}]
		\item[Description] #3
		\item[User Representation] #4 
		\ifthenelse{\equal{#5}{}}{}{\item[Literature] #5}
		\item[Type] #6
		\item[Perspective] #7
		%\ifx #8\empty  #8 \fi
		\item[Characteristics] \ifthenelse{\equal{#8}{}}{n/a}{#8}
		\end{description}
	
	\end{framed}
\end{minipage}
}
\ccCard{CC-1}{Identity \& Lifecycle Commonality}{Nanoentities that belong to the same identity and therefore share a common lifecycle. }{- Entity-Relationship Models \newline - Domain-Driven Design Entities.}{Entity definition in Domain-Driven Design: \newline \textit{Some objects are not defined primarily by their attributes. They represent a thread of identity that	runs through time and often across distinct representations.}\cite{evans2003domain}}{Cohesiveness}{Domain}{}


Cards share the following information:

\begin{description}
	\item[Description] explains the coupling criteria in more detail.
	\item[User Representation] lists concepts or artifacts familiar to the architect that can be used to feed the criteria information into the Service Cutter.
	\item[Literature] references the coupling criteria to descriptions in existing literature.
	\item[Type] Cohesiveness, Compatibility, Constraint or Communication.
	\item[Perspective] Domain, Quality, Infrastructure or Security. 
	\item[Characteristics] can be applied to a nanoentity and are defined for criteria of type \enquote{compatibility}.
\end{description}

\ccCard{CC-2}{Semantic Proximity}{Two nanoentities are semantically proximate when they have a semantic connection given by the business domain. The strongest indicator for semantic proximity is coherent access on nanoentities within the same use case.}{- Coherent access or updates on nanoentities in use cases.\newline - Aggregation or association relationships in an entity-relationship model.}{Chris Richardson on microservice decomposition:\newline \textit{Deciding how to partition a system into a set of services is very much an art but there are number of strategies that can help. One approach is to partition services by verb or use case.}\cite{microserviceIntro}\newline Single Responsibility Principle by Robert Martin:\newline \textit{Gather together the things that change for the same reasons. Separate those things that change for different reasons.}\cite{SRP}}{Cohesiveness}{Domain}{}

\ccCard{CC-3}{Shared Owner}{The same person, role or department is responsible for a group of nanoentities. Service decomposition should try to keep entities with the same responsible owner together while not mixing entities with different responsible instances in one service. }{User defined persons, roles or departments with each containing a group of nanoentities. A nanoentity can only be associated once.}{Conway's law:\newline \textit{Organizations which design systems are constrained to produce designs which are copies of the communication structures of these organizations.}\cite{conway}\newline Single Responsibility Principle by Robert Martin:\newline \textit{Gather together the things that change for the same reasons. Separate those things that change for different reasons. [...] However, as you think about this principle, remember that the reasons for change are people. It is people who request changes. And you don't want to confuse those people, or yourself, by mixing together the code that many different people care about for different reasons.}\cite{SRP}}{Cohesiveness}{Domain}{}

\ccCard{CC-4}{Structural Volatility}{How often change requests need to be implemented affecting a nanoentity's structure.}{Classification of nanoentities in characteristics.}{David Parnas on modular programming: \newline \textit{We propose instead that one begins with a list of difficult design decisions or design decisions which are likely to change. Each module is then designed to hide such a decision from the others.}\cite{parnaDecomposing}}{Compatibility}{Domain}{Often, Normal \textit{(default)}, Rarely}

\ccCard{CC-5}{Latency}{Groups of nanoentities with high performance requirements for a specific user request. These nanoentities should be modelled in the same service to avoid remote calls.}{Use cases with latency requirements. All nanoentities read or written by the same use case belong a group. A nanoentity can belong to multiple groups.}{Design guidelines for application performance by Microsoft: \newline \textit{Minimize round trips to reduce call latency. For example, batch calls together and design coarse-grained services that allow you to perform a single logical operation by using a single round trip.}\cite{performance}}{Cohesiveness}{Quality}{}

\ccCard{CC-6}{Consistency Criticality}{Some data such as financial records loses its value in case of inconsistencies while other data is more tolerant to inconsistencies.}{Classification of nanoentities in characteristics.}{Werner Vogels on consistency requirements: \newline  
\textit{\textbf{Strong consistency}: After the update completes, any subsequent access (by A, B, or C) will return the updated value. \newline \textbf{Weak consistency}: The system does not guarantee that subsequent accesses will return the updated value. A number of conditions need to be met before the value will be returned. The period between the update and the moment when it is guaranteed that any observer will always see the updated value is dubbed the inconsistency window. \newline \textbf{Eventual consistency}: The storage system guarantees that if no new updates are made to the object eventually all accesses will return the last updated value.}\cite{consistency}}{Compatibility}{Quality}{High consistency \textit{(default)}, eventual consistency, weak consistency}

\ccCard{CC-7}{Availability Criticality}{Nanoentities have varying availability constraints. Some are critical while others can be unavailable for some time. As providing high availability comes at a cost, nanoentities classified with different characteristics should not be composed in the same service.}{Classification of nanoentities in characteristics.}{Eoin Woods on availability and resilience: \newline \textit{Getting your availability characteristics wrong can be very expensive. However, increased online availability comes at a cost, whether in terms of more hardware, increased software sophistication, or redundancy in your communications network.}\cite{rozanski2011software} }{Compatibility}{Quality}{Critical, Normal \textit{(default)}, Low}

\ccCard{CC-8}{Content Volatility}{A nanoentity can be classified by its volatility which defines how frequent it is updated. Highly volatile and more stable nanoentities should be composed in different services.}{- Volatility can be calculated from use case definitions if they are equipped with a frequency information. \newline- Nanoentities can be classified by data types to determine the volatility: Master Data (regularly), Reference Data (rarely), Transaction Data (often) and Inventory Data (often) to determine the volatility.}{}{Compatibility}{Quality}{Often, Regularly \textit{(default)}, Rarely}

\ccCard{CC-9}{Consistency Constraint}{A group of nanoentities that have a dependent state and therefore need to be kept consistent to each other. }{An aggregate as defined in domain-driven design.}{Aggregate as defined in domain-driven design by Eric Evans: \newline \textit{A cluster of associated objects that are treated as a unit for the purpose of data changes. External references are restricted to one member of the aggregate, designated as the root. A set of consistency rules applies within the aggregate's boundaries.}\cite{evans2003domain} \newline Udi Dahan on service decomposition: \newline \textit{If modifying the value of one attribute involves changing the value of another, then those two attributes should fall under the responsibility of the same service.}\cite{udiConsistency}}{Constraint}{Quality}{}

A \textit{Consistency Constraint} differs from the \textit{Consistency Criticality} coupling criteria in such a way that the constraint groups a set of nanoentities ensuring their atomic processing. 

For example, a payment and the account balance should be linked using a \textit{Consistency Constraint}. The Service Cutter therefore guarantees that those fields are in the same service. The \textit{Consistency Criticality} criterion is of type compatibility and only separates divergent characteristics.

\ccCard{CC-10}{Mutability}{Immutable information is much simpler to manage in a distributed system than mutable objects. Immutable nanoentities are therefore good candidates for the published language shared between two services. Service decomposition should be done in a way that favors sharing immutable nanoentities over mutable ones. }{Classification of each nanoentity in mutable or immutable.}{Udi Dahan on finding service boundaries:\newline \textit{When you find something immutable, that is an indication that there is some loose coupling between the two sides passing immutable data.}\cite{mutability}}{Communication}{Quality}{}

\ccCard{CC-11}{Storage Similarity}{Storage that is required to persist all instances of a nanoentity.}{The user classifies nanoentities into the given characteristics. The classification is system specific. In one system a nanoentity classified as \textit{huge} might need 1MB, but in another 1GB storage per instance.}{}{Compatibility}{Infrastructure}{Huge, Normal \textit{(default)}, Tiny}

\ccCard{CC-12}{Predefined Service Constraint}{There might be the following reasons why some nanoentities forcefully need to be modeled in the same service: \newline - Technological optimizations \newline - Legacy systems}{User defined service with each containing a group of nanoentities. Each nanoentity can be associated only once.}{}{Constraint}{Infrastructure}{}

\ccCard{CC-13}{Network Traffic Similarity}{Service decomposition has a significant impact on network traffic, depending on which nanoentities are shared between services and how often. Small and less frequently accessed nanoentities are better suited to be shared between services.}{Use cases define the frequency of access on nanoentities. The size of each entity can be determined by CC-11: \textit{Storage Similarity}}{}{Communication}{Infrastructure}{}

\ccCard{CC-14}{Security Contextuality}{A security role is allowed to see or process a group of nanoentities. Mixing security contexts in one service complicates authentication and authorization implementations.}{User defined security roles with each containing a group of nanoentities. A nanoentity can be associated to multiple groups.}{}{Cohesiveness}{Security}{}

\ccCard{CC-15}{Security Criticality}{Criticality of an nanoentity in case of data loss or a privacy violation. Represents the reputational or financial damage when the information is disclosed to unauthorized parties. As high security criticality comes at a cost, nanoentities classified with different characteristics should not be composed in the same service. }{Classification of nanoentities in characteristics}{}{Compatibility}{Security}{Critical, Internal \textit{(default)}, Public}

\ccCard{CC-16}{Security Constraint}{Groups of nanoentities are semantically related but must not reside in the same service in order to satisfy information security requirements. This restriction can be established by an external party such as a certification authority or an internal design team.}{Demilitarized zones or other groups of nanoentities that should be composed to different services.}{}{Constraint}{Security}{}

\clearpage

\section{Decomposition Questionnaire}
\label{sec:decompositionRequirements}

The following questionnaire is based on the coupling criteria and aims to fulfill the principles introduced in Chapter \ref{cha:analysis}. In a good decomposition solution, the answer to all those questions should be \textit{yes}.

\begin{enumerate} 
	\item Does the service cut comply all constraint criteria?
	\item Does the service cut combine as few nanoentities with diverging characteristics into one service as possible?
	\item Does each service depends on as few nanoentities of other services as possible? A use case should cross as few service boundaries as possible. %Do a small amount of use cases require more than one service to be completed?
	\item Are the nanoentities that are part of a published language between services suitable for intra service communication?
	\item Is the coupling between services similar? It is not the size of services that requires homogeneity within the system but the amount of published language between services. 
	\item Are there not too many services? This is called the nanoservice antipattern\cite{nanoservice}.
	\item Are there not too few services? This is a monolithic architecture.
\end{enumerate}

These questions form a checklist to validate a service cut suggested by the Service Cutter. This approach is outlined in Section \ref{sec:suggested-cut-grades}.

\bigskip
After presenting the decomposition model, the next chapter defines the requirements for the prototypically implementation of the Service Cutter.  