
\chapter{Requirements}
\label{cha:requirements}

%TODO: describe the service cutter as a solution finding tool (not a architecture builder) either here or in introduction

This chapter describes the (non-)functional requirements of the target solution and covers the characteristics of its target users. 

The requirements in this chapter are not prioritized. As part of the sprint planning meetings, these requirements are transformed to tasks and prioritized. The description in this chapter is meant to provide a high level overview and establish a common sense between all stakeholders of this project.


\section{Personas}

Our personas are based on ??. % TODO workshops, interviews, discussions?

All users of the system are experienced software architects or developers interested in the educational aspects of the tool.


%TODO other persona Ideas: High level architect, developr and Business analysts

\subsection{Junior Jedi-Master}

Junior has been a fast growing and aspiring developer since he graduated from university with a master's degree in information technology eight years ago. In his new job, Junior finds himself in the role of an architect for a new and promising product that enjoys the support of well known investors. In consequence of his experience in distributed systems, he has been assigned with the task to decompose the system's business model into logical services of which each will be split into multiple separately deployable microservices. 

Junior strongly believes in automation and using every tool available to support and complete his work. For his current project he plans to try the Service Cutter as a foundation and verification of his architectural decisions. 


\subsection{Walter Wisenheimer}

Walter is an architect with many years of experience in the industry and has built numerous systems already. Walter has seen many automation concepts and tools failing their goals. In Walter's view a natural and obvious outcome -- an architect's world is too complex to model and automate in a system or algorithm. 

After a longer discussion with an over-motivated greenhorn who recently joined his company, Walter starts to see the benefits of the well structured format the Service Cutter organizes architecturally relevant information. Using the tool to structure and document his systems characteristics might be of some benefit as currently most of this information is only known to himself. 

Walter decided to try the Service Cutter to structure the information for the current project he is working on.

%\subsection{Charles Consultant}

%TODO necessary?

\subsection{Stan Student}

Stan studies Computer Science and as part of his class in Service oriented Software Architecture he is supposed to design a set of services for the ZooKeeper domain. The Service Cutter guides him through the important decisions by asking a set of questions and presents him a set of possible service cuts. 

Being overwhelmed of all the data requested by the Service Cutter he would like to configure the tool to only focus on the data he got provided in his exercises. 

Stan then discusses the advantages and disadvantages of the presented options with his fellow students. He furthermore asks his Professor about the to him unknown criteria the Service Cutter requested and why that information might have an impact on service orientation. 


\subsection{Tom Tutor}

Tom wants to introduce his students to the software architecture craft. He uses the Service Cutter to visualize the different ways of distribute data into services during his lectures. By changing the calculation parameters he can demonstrate that software architecture mostly depends on the context of the requirement. The same problem might have different solutions in varying circumstances.


\section{Functional Requirements}

The Service Cutter faces the functional requirements presented in the following
section.

%TODO machen wir user stories?

%TODO: each data field needs a master who is the only one doing writes.

%TODO low prio: Add relations between coupling criteria. examples: High volatility and high security is dangerous. Or high criticality and high change management. The Service Cutter could generate warnings if theses things are put together in the same service or either consider theses r


%TODO low prio: Show warnings as part of the output should a predefined service have a significant negative effect!

\subsection{Data Fields and Entities}

A set of data fields can either be defined in the application or imported using a predefined data format. A data field consists of it's name and a combination of connections to other fields, referred to as Coupling Criteria.

The process of putting fields in relation to each other should use as many familiar concepts as possible. The concept of an entity - a set of fields sharing a commdon identity and lifecycle\cite[p.13]{evans2014domain} - is a widely used notion and is therefore used by the architect to model the data and then mapped to the Coupling Criteria \enquote{Shares Lifecycle} and \enquote{Common Identity} by the system.

\subsection{Coupling Criteria}

Almost all data fields are related to other fields. The concept of a Coupling Criterion is used to characterise such connections.

A list of relevant Coupling Criteria was identified in a workshop with Zühlke. The following Criteria are supported by the system:

\begin{itemize}
	\item Shares Lifecycle
	\item Common Identity
	\item Inheritance
	\item Volatility
	\item Security
	\item Resilience
	\item Use Case
\end{itemize}

%TODO liste ergänzen nach workshop!

\subsection{Service Boundaries}

Once the user has added all data fields and coupling criteria he can calculate the suggested service boundaries. An algorithm groups fields in a way that minimizes coupling between the groups while maximizing the cohesion inside a group.

The calculation can be parametrised using weights assigned to coupling criteria. In this way different service boundaries can be calculated for different requirements. For example in an application that processed financial data, security might be more important than Use Cases. In a different context, e.g. for an application that needs to support high volumes of data, volatility and resilience might be a primary focus.

\subsection{Service Coupling}

Some Coupling Criteria will likely span across service boundaries. The architect uses the graphical representation of the Service Cutter to inspect the coupling. Using this information he might change the calculation parameters to achieve a different service cut.

Service Coupling maps to the Published Language as described by Evans \cite[p.375]{evans2003domain}.

% link zu published language von DDD?

% TODO:  Quantification is configurable – but the system provides a good default
\section{Nonfunctional Requirements}
\label{sec:nonfunctionalRequirements}

The following non-functional requirements should be satisfied by the Service Cutter.

\subsection{Usability}
\label{sec:usability}

A software architect should be able to use the software without any training. All controls are clearly named and, where appropriate, documented using an inline user manual including representative samples.

The user gets well introduced to the important concepts of the Service Cutter. These concepts include:

\begin{enumerate}
	\item A Service and service decomposition
	\item A system's model (nanoentities)
	\item Coupling criteria, their meaning and decomposition impact
	\item User representations and their impact on coupling criteria
	\item Characteristics values, coupling criteria priorities and final ratings.
\end{enumerate}

Often used configuration like coupling criteria priorities should be shown directly to the user to encourage its usage. More advanced configuration like the weights of coupling criteria variants should be accessible for the user but do not need to be shown directly in a standard workflow. 

Up to 2000 nanoentities and 200 entities need to be manageable without losing track.

\subsection{Simplicity}

A simple system analysis can be achieved with not more than 5 clicks. All steps are provided with useful defaults that can be changed.

\subsection{Performance}

All regular user interactions should not take more than one second.

Calculations of service cuts should meet the following conditions assuming a data set of 100 entities and 2000 nanoentities.

\begin{itemize}
\item Calculations that are used once per day should take less than 10 minutes.
\item Calculations that are used once per minute should take less than 5 seconds.
\end{itemize}

\subsection{Monitoring, Logging, Deployment, Availability}

The scope of this thesis is to prototypically implement the Service Cutter. Operational aspects only need to be covered on a basic level:

\begin{itemize}
	\item Log files should be written using SLF4J\cite{slf4j}.
	\item Deployment should be provided with Docker\cite{docker} containers.
\end{itemize} 

\subsection{Fault Tolerance}

As the prototype does not need to handle ever possible use case or unexpected input, a common error handling needs to be built into the application and ensures that operations continue even in case of unexpected input or state.

\subsection{Maintainability}

In case of a success the prototype might be the basis for further development or other thesis projects. It should be built in clearly separated modules (or even remote services) to ensure good maintainability. At least the following modules need to be clearly separated:

%TODO thesis plural

\begin{enumerate}
	\item Internal representation of coupling criteria
	\item Data of a users system (nanoentities, use case definition etc.) 
	\item Decomposition algorithm (Solver)
	\item User interface
\end{enumerate}

If one or multiple of the modules are implemented as physical services the remote \gls{API} needs to be implemented as RESTful HTTP interface. 

The application should leverage existing open source frameworks or libraries wherever possible to ensure a minimal maintenance effort.

\subsection{State-of-the-Art Technology}

To support further development of the Service Cutter the prototype needs to be implemented in state-of-the-art technology. Our industry partner Z\"uhlke has suggested the following technology stack:

\begin{enumerate}
	\item Java Spring\cite{spring} as a base framework. 
	\item JHipster\cite{jhipster} with Spring Boot\cite{springboot} for project setup.
	\item AngularJS\cite{angularjs} and Bootstrap\cite{bootstrap} for the user interface.
	\item Docker\cite{docker} for container and deployment configuration and handling. 
\end{enumerate}

\subsection{License}

All involved parties decided to release the Service Cutter under the terms of the Apache 2.0 open source license.


\bigskip

After defining all (non-)functional requirements, the next Chapter outlines design and implementation steps which were taken to satisfy these requirements.

