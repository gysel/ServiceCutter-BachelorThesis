\chapter{Introduction}

This chapter introduces the project's goals, scope and context. The original project definition, signed at the beginning of project, is documented in Appendix \ref{appendix:projectDefinition}.

\section{Hypothesis}

D. L. Parnas published a paper titled \textit{On the Criteria To Be Used in Decomposing Systems into Modules}\cite{parnaDecomposing} in 1972. Since then, decomposition of software systems has become an important area in the field of software engineering. As systems grew more complex, software engineers started to distribute modules over computer networks and hence called them services. Architectural styles like Software Oriented Architecture (SOA) have been introduced to tackle many challenges of such distributed systems.

Nevertheless, even with microservices, the latest incarnation of service orientation, decomposition is more described as an art than a structured discipline. C. Richardson writes in his popular introduction to microservices on \gls{infoq}:

\begin{quote}
	\textit{Deciding how to partition a system into a set of services is very much an art but there are number of strategies that can help. One approach is to partition services by verb or use case.}\cite{richardson2014microservices}
\end{quote}

We consider the described strategies as suitable approaches to service decomposition. However, we assume that service decomposition can be approached in more structured way. This leads us to our first hypothesis:

\begin{quote}
	\textit{The driving forces for service decomposition of a software system can be assembled in a comprehensive criteria catalog.}
\end{quote}

\clearpage

To validate this first hypothesis, we created a comprehensive but not conclusive catalog of coupling criteria. Taking this structured approach to service decomposition a step further, we formulated a second hypothesis:

\begin{quote}
	\textit{Based on the criteria catalog, a system's specification artifacts can be processed in a software to optimize loose coupling between services and high cohesion within services in a structured and automated way.}
\end{quote}

To validate this second hypothesis, we developed a prototype based on the criteria catalog. This tool, hence called the \enquote{Service Cutter}, analyzes a system's specification and suggests candidate service cuts in order to optimize loose coupling between services and high cohesion within services. A system's specification contains an entity-relation model, use cases, and further artifacts as illustrated in Figure \ref{fig:serviceCutterIO}.

\begin{figure}[H]
	\begin{center}
		\includegraphics[scale=1.1]{diagrams/systemContextDiagram.pdf}
	\end{center}
	\caption{Input and output of the Service Cutter.}
	\label{fig:serviceCutterIO}
\end{figure}

The Service Cutter's goal is to assist and advise a software architect or developer in his design decisions regarding service decomposition. The architect needs to assess the candidate service cuts and compare them with his expectations. The Service Cutter's mission is accomplished, if the architect's expectations are verified or unexpected but reasonable candidate cuts challenge his preoccupations. 

\section{Project Scope}

This section describes the scope and boundaries of this thesis. It first defines some of the relevant terms used throughout this document.

A \textit{system} refers to a software application whose architecture needs to be decomposed.

A \textit{service} can be seen as a module providing a remote \gls{API} to communicate with other services. The term is explained in more detail in Chapter \ref{cha:analysis}.

\textit{Service decomposition} refers to splitting a system's functionality and data into services. While we focus on service decomposition, most of the concepts are also true for non-distributed systems where a software internally is decomposed into modules. 

Before a system can be decomposed, its functional and non-functional requirement need to be analyzed and specified in an entity-relationship model, use cases, and other artifacts. Based on these specifications the system can be decomposed into services. These are later implemented and connected using intra service communication. Figure \ref{fig:context} illustrates this process.

\begin{figure}[H]
	\begin{center}
		\includegraphics[scale=1.4]{diagrams/context.pdf}
	\end{center}
	\caption{The thesis in the context of system development.}
	\label{fig:context}
\end{figure}

Our thesis focuses solely on service decomposition. Consequently, the following areas are not in scope:

\begin{itemize}
	\item Requirements engineering and system specification need to be done before a system can be analyzed with the Service Cutter.
	\item Intra service communication is not in scope of this thesis. S. Newman documents in his book \textit{Building Microservices}\cite{newman2015building} multiple popular ways for intra service communication like \gls{RPC}, RESTful HTTP services or asynchronous event-based collaboration. Decomposition only defines \textit{what} is communicated but now \textit{how}. 
	\item Composing multiple services into workflows or business processes using notations like \gls{BPMN} is not in scope.
	\item Service decomposition tries to minimize coupling between services. Tactics like caches or \gls{CQRS}, which try to lower the consequences of coupling introduced by decomposition, are not analyzed. 
\end{itemize}


\section{Context and Influences}

The ideas and concepts presented are influenced by the work of many others. We reused and embodied existing concepts to our structured way of service decomposition. 

\subsection{Service Oriented Architecture}

It was during a course titled \textit{Advanced Distributed Systems Design using SOA \& DDD} by Udi Dahan where our initial idea to assist service decomposition with an automated approach emerged. Dahan is the founder of NServiceBus\cite{nservicebus}, the most popular service bus for \gls{dotnet} and a well known \gls{SOA} and \gls{DDD} expert. The approach to tackle service decomposition from the 4+1 View Model\cite{fourPlusOne} is inspired by him. Approaching decomposition on the basis of data fields, or the later in the document introduced nanoentites, is motivated by his course.

Further \gls{SOA} influences are provided by our supervisor throughout the project and during his course \textit{Application Architecture} at \gls{HSR}.

\subsection{Microservices}

In recent years, microservices substituted \gls{SOA} as the trending architectural style, but can be seen as a new incarnation of the service oriented approach. Valuable concepts like service definitions or decomposition criteria are inspired by leading evangelists in this area such as Martin Fowler, Sam Newman, and Chris Richardson. 

\subsection{Domain-Driven Design}

Nevertheless, decomposition is not solely a problem in distributed systems. Eric Evans introduced in his book \textit{Domain-Driven Design: Tackling Complexity in the Heart of Software}\cite{evans2003domain} a collection of patterns to handle decomposition complexity. Especially the patterns \textit{Aggregate}, \textit{Entity}, \textit{Published Language} and \textit{Bounded Context} are integrated in our approach and serve as input or output of the Service Cutter.

\section{Market Overview}

We were not able to find projects that try to structure and automatically assist service decomposition. Nevertheless, there are several methodologies and decomposition tools tackling some of the relevant challenges. 

\subsection{Software Methodologies}

The already introduced approaches \gls{SOA}, microservices, and \gls{DDD} discuss some decomposition criteria but do not provide a comprehensive criteria catalog. 

Other approaches tackle related problems but are focused on different layers of software development. \gls{OOAD} focuses more on abstractions like classes and object instances. We integrated \gls{OOAD} artifacts like the \gls{ERM} as part of the system specification given to the Service Cutter as input. \gls{BPM} lays an abstraction layer above services, focusing on business processes and therefore the usage of services rather than their identification and specification. A detailed analysis on the correlation of \gls{OOAD} and \gls{BPM} with service orientation was published by IBM\cite{zimmermann2004elements}. Service oriented modeling approaches like \gls{SOMA}\cite{arsanjani2004service} target similar questions as this thesis but do not provide detailed descriptions of decomposition approaches. \gls{SOMA} suggests decomposition by use cases which resembles our decomposition criterion \textit{Semantic Proximity} introduced in Chapter \ref{cha:decomposition}.

\subsection{Decomposition Tools}

Kenny Bastani suggests a graph based analysis to decompose monolithic software into microservices\cite{bastani}. In his decomposition approach, he focuses on dependencies from user stories to RESTfull HTTP resources. He uses Neo4J GraphGist\cite{graphGist} to visualize the dependencies but does not run any automated analysis on the graph. 

The barrio eclipse plugin\cite{dietrich2008cluster} analyzes dependencies based on Java source code. It suggests a candidate package structure by leveraging the Girvan-Newman clustering algorithm. The tool has been published as part of a student's project of the Massey University, New Zealand.

\bigskip

After introducing our hypothesis's and the broader context of service decomposition, the next Chapter analyzes the definition of a service and service decomposition principles in more detail. 

