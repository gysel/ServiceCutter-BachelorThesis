\chapter{Introduction}

D. L. Parnas published a paper titled \textit{On the Criteria To Be Used in Decomposing Systems into Modules}\cite{parnaDecomposing} in 1972. Since then, decomposition of software systems has become an important area in the field of software engineering. As systems grew more complex, software engineers started to distribute modules over a network and hence called them services. Architectural styles like Software Oriented Architecture (SOA) have been introduced to tackle the many challenges such distributed systems create. 

Nevertheless, even with microservices, the latest incarnation of service orientation, decomposition is more described as an art then a structured discipline. C. Richardson writes in his popular introduction to microservices on InfoQ:

\begin{quote}
	\textit{Deciding how to partition a system into a set of services is very much an art but there are number of strategies that can help. One approach is to partition services by verb or use case.}\cite{richardson2014microservices}
\end{quote}

As we acknowledge the suitability of the described strategies, we do believe that there is a more structured way to service decomposition. This leads us to our first hypothesis:

\begin{quote}
	\textit{The driving factors for service cuts in a software system can be compiled into a comprehensive criteria catalog.}
\end{quote}

To take the structured approach to service decomposition a step further, we formulated a second hypothesis:

\begin{quote}
	\textit{The data of the criteria catalog can be embodied in software in order to optimzie loose coupling between and high cohesion within services in a strcutured and automated way.}
\end{quote}

To proof our hypothesis's 




During a workshop titled  led by Udi Dahan, 


Starting with Microservices, what news they bring (deployability) and what old, underlying problem they try to solve (decomposition)

These: There is more to service decomposition than "Belongs to bigger concept defined by noun or verb(Chris Richardson)
=> Goal/Idea

- Identify coupling criteria and constraints after which service 
decomposition should be done. 

- Build a tool that takes nanoentities and applied coupling criteria and constraints as input and produces suggestions on how to split the system in different parts. 

=> thesen

=> identification, specification, implementation

\section{Project Definition}



\section{Scope}

\textbf{Notes only, not to be reviewed}

Identification \& Implementation not in scope

Focus only on data, not on logic. (??)

No focus on ways to handle or soften coupling (async, interfaces, messaging, CAP, REST, RPC, etc.)

Begriffe: Module, Service, Microservice, Komponente, DDD, Bounded Context, Entity, Aggregate, Classes, Packages

Service
Service Boundary Recommondations
Service/BC > Microservices 
AppArch 3 Ebenen von Services
Drei definitionen von SOMA/Services, User / Architect / Developer