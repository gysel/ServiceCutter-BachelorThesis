\chapter{Introduction}

D. L. Parnas published a paper titled \textit{On the Criteria To Be Used in Decomposing Systems into Modules}\cite{parnaDecomposing} in 1972. Since then, the challenge of decomposing a system into modules has been widely discussed and many principles like the \textit{Single Responsibility Principle} and metrics like \textit{cohesion} and \textit{coupling} emerged. 

With 

Starting with Microservices, what news they bring (deployability) and what old, underlying problem they try to solve (decomposition)

These: There is more to service decomposition than "Belongs to bigger concept defined by noun or verb(Chris Richardson)
=> Goal/Idea

- Identify coupling criteria and constraints after which service 
decomposition should be done. 

- Build a tool that takes nanoentities and applied coupling criteria and constraints as input and produces suggestions on how to split the system in different parts. 

=> thesen

=> identification, specification, implementation

\section{Project Definition}



\section{Scope}

\textbf{Notes only, not to be reviewed}

Identification \& Implementation not in scope

Focus only on data, not on logic. (??)

No focus on ways to handle or soften coupling (async, interfaces, messaging, CAP, REST, RPC, etc.)

Begriffe: Module, Service, Microservice, Komponente, DDD, Bounded Context, Entity, Aggregate, Classes, Packages

Service
Service Boundary Recommondations
Service/BC > Microservices 
AppArch 3 Ebenen von Services
Drei definitionen von SOMA/Services, User / Architect / Developer