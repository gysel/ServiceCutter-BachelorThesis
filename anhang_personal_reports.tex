\chapter{Personal Reports}

\section{Michael Gysel}


\section{Lukas Kölbener}

In July this year I attended a course called \textit{Advanced Distributed Systems Design using SOA \& DDD} by Udi Dahan in Oslo. As part of this course we conducted a service decomposition exercise where we assigned data fields to services based on use cases and characteristics such as data volatility. During this exercise, I had the idea to automate this process or at least assist it by software and algorithms. 

Half a year later I gratefully look back on a very interesting, instructive and intense bachelor thesis in which Michael Gysel and me tried to realize this idea. 

The main activities in this project were:

\begin{itemize}
	\item Architectural discussions and research.
	\item Developing of a web based software using cutting-edge technologies.
	\item Analyzing and integrating algorithms.
	\item Writing in an extensive paper in English.
\end{itemize}

I am interested in all of these fields and was able to gain many more insights, so that I enjoyed working on the thesis almost all the time. A personal highlight was the moment during Sprint 3 when we first assessed the Trading System. The Service Cutter provided the service cuts exactly as expected. Until then I was never sure but only hoped that the automated service decomposition is possible at all. 

The teamwork with Michael Gysel was excellent as we are experienced in conducting projects together and have complementary strengths and skills. Quite often we did not share the same opinion but found compromises integrating the best of each approach. I am convinced The quality of this project is the direct result of such discussions. Working together with our supervisor and industry partner was constructive, insightful and encouraging at all times. 

The biggest challenge was to keep track and overview over a very broad field and ambitious goals. We often had to stop working on parts we would have liked to invest more in order focus on the main goals. To satisfy a variety of requirements from different stakeholders taught us to clearly prioritize tasks. 

To summarize, I am content with the created prototype, this documentation and the personal experience and learnings I have earned throughout this project. 

