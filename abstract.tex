\chapter{Abstract}

Decomposing a software system into smaller parts has been an important discipline in our industry for many decades. With the rise of distributed systems, it has become more important to split a system into low coupled and high cohesive parts. Architectural styles like Service Oriented Architecture (SOA) and currently Microservices tackle the many challenges of such systems but remain vague on the art of how to decompose a system into services.

With the help of our industry partner Zühlke, our supervisor Prof. Dr. Olaf Zimmermann, and existing literature, we introduce a structured way to service decomposition by providing a comprehensive coupling criteria catalog.

We embodied these coupling criteria in the Service Cutter, a prototype that extracts coupling information out of well-known concepts like domain models and use cases. Using this information, the Service Cutter’s mission is to produce service cuts to assist an architect’s decomposition decisions. 

A scoring system is defined to automatically interpret coupling data. By employing a weighted, undirected graph and clustering algorithms the Service Cutter produces service cuts that minimize coupling between services while ensuring high cohesion within a service. 

Tests with two sample projects not only met our expectations but produced reasonable service cuts that have not been considered before. 

With a more structured way to service decomposition, the Service Cutter demonstrate that automated decision assistance is a promising way. The thesis lays the foundation for further projects in this area. 
