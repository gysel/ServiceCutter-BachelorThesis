
\section{Personas}

The personas are inspired by a series of discussions in meetings and workshops with our stakeholders.

\subsection{Junior Jedi-Master}

Junior has been a fast learning and aspiring developer since he graduated from university with a master's degree in information technology eight years ago. In his new job, Junior finds himself in the role of an architect for a new and promising product that enjoys the support of well known investors. In consequence of his experience in distributed systems, he has been assigned with the task to decompose the system's business model into logical services of which each will be split into multiple separately deployable microservices. 

Junior strongly believes in automation and using every tool available to support and complete his work. For his current project he plans to try the Service Cutter as a foundation and verification of his architectural decisions. 


\subsection{Walter Wisenheimer}

Walter is an architect with many years of experience in the industry and has built numerous systems already. Walter has seen many automation concepts and tools failing their goals. In Walter's view a natural and obvious outcome -- an architect's world is too complex to model and automate in a system or algorithm. 

After a longer discussion with an very motivated junior developer who recently joined his company, Walter starts to see the benefits of the well structured format the Service Cutter organizes architecturally relevant information. Using the tool to structure and document his systems characteristics might be of benefit as currently a lot of his precious knowledge remains tacit\cite{zimmermann2015architectural}. 

Walter decided to try the Service Cutter to structure the information for the current project he is working on.


\subsection{Stan Student}

Stan studies Computer Science and as part of his class in service oriented software architecture he is supposed to design a set of services for the Cargo Tracking\cite{dddGithub} domain. The Service Cutter guides him through the important decisions by asking a set of questions and presents him a set of possible service cuts. 

Being overwhelmed of all the data requested by the Service Cutter, he would like to configure the tool to only focus on the data he got provided in his exercises. 

Stan then discusses the advantages and disadvantages of the presented options with his fellow students. He furthermore asks his Professor about the to him unknown criteria the Service Cutter requested and why that information might have an impact on service decompostion. 


\subsection{Tom Tutor}

Tom wants to introduce his students to the software architecture craft. He uses the Service Cutter to visualize the different ways of distributing data into services during his lectures. By changing the calculation parameters he can demonstrate that software architecture mostly depends on the context of the requirement. The same problem might have different solutions in varying circumstances.

\subsection{Eddie Enterprise}

Eddie is an enterprise architect employed by a large software consulting company. He usually works for a couple of months on a project. His customers expect from him that he influences important decisions even beyond his assignment. To achieve this, he decided together with the local project team that every decision they take has to be documented using a structured approach. He would like to utilize the Service Cutter's approach to ensure that all important aspects of coupling and cohesion are considered as part of the service decomposition.
