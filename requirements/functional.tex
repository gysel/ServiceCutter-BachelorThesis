\section{Functional Requirements}
\label{sec:functionalRequirements}

The Service Cutter faces the functional requirements presented in this
section.

%TODO low prio: Show warnings as part of the output should a predefined service have a significant negative effect!

\subsection{Coupling Criteria}

The criteria of type \textit{Cohesiveness}, \textit{Compatibiltiy} and \textit{Constraints} must be supported. The Service Cutter needs to rate for each criteria which nanoentities need be placed in one service and which need to be separated over multiple services. Within this rating, every coupling criterion is equally respected.

Criteria of type \textit{Communication} do not describe the need to merge or separate nanoentities but characterize nanoentities suitable for inter service communication and are handled with a lower priority than other criteria. 

\subsection{User Representations}

To achieve better usability, the user does not need to now the exact definitions of the coupling criteria or their internal structure. He can use well known concepts called \textit{user representations} to describe his system, from which the relevant nanoentities and coupling criteria data will be extracted by the Service Cutter. At least the following User Representation must be supported by the system:

\begin{description}
	\item[Entity Relationship Diagram] containing \glspl{entity} and their relations to each other. Each entity contains a list of nanoentities building the basis for a systems analysis.
	\item[Use Case] containing at least information about all nanoentities read or written in a particular case. 
	\item[Categorization] of the nanoentities in different characteristics per compatibility coupling criteria. 
\end{description}

\subsection{Priorities}

As every coupling criteria is respected equally in the rating process, the user needs an additional way to prioritize criteria according to its system characteristics. For an application processing financial data, security might be more important than network traffic. For an application that needs to support high volumes of data, volatility and resilience might be a primary focus.

The system should allow to change priorities for all supported coupling criteria while providing reasonable defaults. 

\subsection{Candidate Service Cuts}

Based on the input provided by user representations and the defined criteria priorities, the Service Cutter produces a set of candidate service cuts for the analyzed system. The definition of a good service cut is elaborated in Section \ref{sec:decompositionRequirements}

A candidate service cut can be exported in a machine-readable format.

\subsection{Visualize Published Language}

If relevant user input (e.g. use case definitions) is available, the Service Cutter is able to tell which service depends on which nanoentities of other services. These dependencies need to be visualized. All nanoentities shared between different services make up the published language of the system and need to be visualized as such.

\subsubsection{Identify Hard Architectural Decisions}
\label{subsec:identifyHardADs}

The rating of two solutions might be similar so that is not clear which variant is the better one according to the users priorities. Such cases indicate difficult design decisions the architect has to take. Presenting these cases and the impact they have on each coupling criteria, the Service Cutter provides great support for the architect in identifying and taking architectural decisions.

