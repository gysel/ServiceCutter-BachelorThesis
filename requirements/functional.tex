\section{Functional Requirements}


\subsection{Data Fields and Entities}

A set of data fields can either be defined in the application or imported using a predefined data format. A data field consists of it's name and a combination of connections to other fields, referred to as Coupling Criteria.

Software Engineers often use entities to create data models. Therefore the concept of entities is used to group data fields in the graphical editing process.

\subsection{Coupling Criteria}

All data fields are related to other fields. The concept of a Coupling Criterion is used to characterise such connections.

The most common Coupling Criterion is \textit{Same Entity} which is commonly modeled using Entities. %TODO cite https://en.wikipedia.org/wiki/Entity%E2%80%93relationship_model

Others include:

\begin{itemize}
	\item Inheritance
	\item Volatility
	\item Security
	\item Resilience
	\item Use Case
\end{itemize}


\subsection{Service Boundaries}

Once the user has added all data fields and coupling criteria he can calculate the suggested service boundaries. An algorithm groups fields in a way that minimizes coupling between the groups while maximizing the cohesion inside a group.

The calculation can be parametrised using weights assigned to coupling criteria. In this way different service boundaries can be calculated for different requirements. For example in an application that processed financial data, security might be more important than Use Cases. In a different context, e.g. for an application that needs to support high volumes of data, volatility and resilience might be a primary focus.