\section{Functional Requirements}

The Service Cutter faces the functional requirements presented in this
section.

%TODO low prio: Show warnings as part of the output should a predefined service have a significant negative effect!

\subsection{Service Cutter Process}

The Service Cutter's analysis and suggestions is based on nanoentities. In a first step, nanoentities are inserted into the Service Cutter and define a system to be analyzed. The user then defines meta data on the nanoentities and their relations in a second step. In the third step, the user runs the decomposition process to receive suggestions of how the nanoentities could be split into Services. He finally analyzes the suggestion and compares it with his own expectations. 

\subsubsection{Step 1: Model Initialization}

In this step a model representing a system to be analyzed is created within the Service Cutter. The user defines all nanoentities relevant in his domain. 

In a first approach, defining these nanoentities should be possible either as a list of nanoentities or parsed from an \gls{ERM} defined in a self defined JSON representation.

\subsubsection{Step 2: Describing the System based on Coupling Criteria}

Almost all nanoentities are related in some way to other nanoentities. The concept of a Coupling Criterion described in section \ref{sec:couplingCriteria} is used to characterize such relations. In this step the user needs to enhance the model with meta information about the nanoentities' characteristics and relations to each other. 

To achieve better usability, the user does not need to now the exact definitions of the Coupling Criteria or their internal structure. He can use well known concepts called \textit{user representations} to describe his system, from which the relevant Coupling Criteria data will be extracted by the Service Cutter. At least the following User Representation should be supported by the system:

\begin{description}
	\item[Use Case] containing at least information about all nanoentities read or written within the particular case. 
		\item[Entity Relationship Diagram] containing \glspl{entity} and their relations to each other. 
	\item[Categorization] of the nanoentities in different characteristics per compatibility coupling criteria. 
\end{description}

Importing the user representation should either be done over a RESTful HTTP API in a self defined JSON format or using a wizard within the Service Cutter's \gls{UI}.

\subsubsection{Step 3: Decomposition}

Once the user has added all nanoentities and Coupling Criteria he can start the decomposition process. An algorithm groups nanoentities in a way that a good decomposition solution as described in \ref{sec:decompositionRequirements} is found. 

The algorithm will then present one or multiple solution suggestions to the user.

The calculation can be parametrized using priorities assigned to Coupling Criteria. In this way different service boundaries can be calculated for different requirements. For example in an application processing financial data, security might be more important than network traffic. In a different context, e.g. for an application that needs to support high volumes of data, volatility and resilience might be a primary focus.

A deterministic algorithm should be favored as it allows reproducibility of the result. For a given data set the Service Cutter should always suggest the same candidate cuts. A non deterministic algorithm may be chosen if it produces noticeably better results.

\subsubsection{Step 4: Working with Solution Suggestions}

(this might not be in scope of this thesis, should we describe this in requirements or in future work?)

Once solution suggestions have been presented, the user needs to interpret them and compare it with his own expectations. There are multiple ways the provided data could be further used in the process of a software engineering project:

\subsubsection{Improve Understanding the own System or Domain}

It is important that the calculated data is presented to the user in a way for him to understand why the suggested solution has been selected. By visualizing user representation like use cases, the architect gets a better understanding of how his input and definitions affect the suggested architecture. 

\subsubsection{Visualize Published Language}

If relevant user input (e.g. use case definitions) is available, the Service Cutter is able to tell which Service depends on which nanoentities of other Services. This dependencies need to be visualized in the solution presentation. All nanoentities together that have dependencies from external Services make up the published language of the system and should be visualized as such.

\subsubsection{Adjust Suggested Solutions}

The architect is able to adjust the solution by moving a nanoentity from one to another Service. He instantly sees the impact of his adjustment by the rating of each Coupling Criteria.

\subsubsection{Identify hard Architectural Decisions}

The rating of two solutions might be similar so that is not clear which variant is the better one according to the users priorities. Such cases indicate difficult design decisions the architect has to take. Presenting these cases and the impact they have on each Coupling Criteria, the Service Cutter provides great support for the architect in identifying and taking architectural decisions.

\subsubsection{Document Architectural Decisions}

\begin{quote}
	Architectural decisions capture key design issues and the rationale behind chosen solutions.\cite{zioAD}
\end{quote}

Documenting architectural decisions is a significant documentation artifact of every long-term software project. In order to retrace architectural decisions taken with help of the Service Cutter, important solutions enhanced with discussion notes can be saved persistently. These notes could have one of the following forms:

\begin{enumerate}
	\item Free text
	\item Y-Template\cite{y-template}
	\item ...
\end{enumerate}

\subsubsection{Use Solution as Fundament for Working Software}

If the solution is good enough to be used as an initial architecture of a system, it can be used to built the suggested Services. As the Service Cutter has sophisticated information about the usage of nanoentities in use cases, it is able to generate the APIs used to communicate between Services. 

Depending on the communication layer used to communicate between Services these APIs look differently. In a messaging based system the Service Cutter generates a set of messages or events needed to communicate between Services. When Services interact by RESTful HTTP interfaces, the Service Cutter generates Swagger\cite{swagger} API definitions for the resources which need to be part of the published language\cite[p.375]{evans2003domain} of the system. 


\subsection{Algorithm Parametrization}

The algorithm uses numbers to quantify characteristics of a compatibility Coupling Criteria. An example for such characteristics is:

TODO: Screenshot Resilience doc from web

Every characteristic has a number from which the Service Cutter calculates distances which are used to apply a penalty to a solution that puts nanoentities with different charcteristics in the same Service. 

For a normal usage, the Service Cutter provides reasonable defaults which have been tested with example projects. For advanced usage the Service Cutter provides an interface to change these internal weights in order to adjust the algorithm interpretation of the input to a systems characteristics. 

\subsection{Relationships between Coupling Criteria}

To keep the decomposition simple, relationships between Coupling Criteria are not taken into account. Nevertheless a relationship like the following has significant impact on how a system is modeled:

\textit{Nanoentities with high security criticality should not be placed in the same service as nanoentities with low consistency requirements.}

Even if these relationships are not considered during decomposition, the Service Cutter should present a warning to the user whenever a critical relationship in a solution appears.
