\section{Nonfunctional Requirements}

The following non-functional requirements should be satisfied by the Service Cutter.

\subsection{Deployment}

No software installation is required. The latest version of any popular internet browser (Chrome, Firefox, Internet Explorer) is sufficient.

web app, no install.
java, docker

\subsection{Usability}
\label{sec:usability}

A software architect should be able to use the software without any training. All controls are clearly named and, where appropriate, documented using an inline user manual.

Up to 500 data fields and 50 entities are manageable without losing track.

The base layout is responsive and adapts to smaller screens such as smartphones. However the tool is mostly used on devices such as laptops and the controls are therefore optimized for use on screens that are at least 15 inches wide and used with a mouse and a keyboard.

\subsection{Simplicity}

A simple workflow can be achieved with a few clicks. All steps are provided with useful defaults that can be changed.

\subsection{Information Security}

\subsection{Separation of Concerns}

\subsection{Performance}

All regular user interactions should not take more than one second. The service boundary calculation may take up to 10 seconds depending on the data model size. The benchmark is the upper limit outlined in section \ref{sec:usability}.

\subsection{Monitoring}

\subsection{Availability and Fault Tolerance}

\subsection{Maintainability}

\subsection{Integrity Tests, Audits, Logs}

\subsection{License}

No GPL?