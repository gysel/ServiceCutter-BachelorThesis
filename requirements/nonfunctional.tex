\section{Nonfunctional Requirements}

The following non-functional requirements should be satisfied by the Service Cutter.

\subsection{Deployment}

No software installation is required. The latest version of any popular internet browser (Chrome, Firefox, Internet Explorer) is sufficient.

On the server side Java is used to implement the business logic. An appropriate storage and web server technology will be evaluated as part of the design phase. The deployment of the different components is managed using \gls{docker} containers.

\subsection{Usability}
\label{sec:usability}

A software architect should be able to use the software without any training. All controls are clearly named and, where appropriate, documented using an inline user manual.

Up to 500 data fields and 50 entities are manageable without losing track.

The base layout is responsive and adapts to smaller screens such as smartphones. However the tool is mostly used on devices such as laptops and the controls are therefore optimized for use on screens that are at least 15 inches wide and used with a mouse and a keyboard.

\subsection{Simplicity}

A simple workflow can be achieved with a few clicks. All steps are provided with useful defaults that can be changed.

\subsection{Information Security}

The web application is secured using an authentication and authorization implementation. Any other internal components such as the database or web services are hidden behind the servers firewall and therefore do not need any special security measures.

The uploaded data models are initially shared amongst all registered users.

\subsection{Separation of Concerns}

The software is structured in a way that separates the application into several concerns running in separated containers that are managed individually.

\subsection{Performance}

All regular user interactions should not take more than one second. The service boundary calculation may take up to 10 seconds depending on the data model size. The benchmark is the upper limit outlined in section \ref{sec:usability}.

\subsection{Monitoring}

The software is not business critical and no explicit monitoring is required.

\subsection{Availability and Fault Tolerance}

The software does not implement any high availability measures. A common error handling is built into the application and ensures that operations continue even in case of unexpected input or state.

\subsection{Maintainability}

The application should leverage existing open source frameworks or libraries whereever possible to ensure a minimal maintenance effort.

\subsection{Logs}

Log files should be written using SLF4J\cite{slf4j} provided by Spring Boot\cite{springboot}.

\subsection{License}

No GPL? to be discussed %TODO