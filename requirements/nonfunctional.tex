\section{Non-Functional Requirements}
\label{sec:nonfunctionalRequirements}

The following non-functional requirements should be satisfied by the Service Cutter.

\subsection{Usability}
\label{sec:usability}

A software architect should be able to use the software without any training. All controls are clearly named and, where appropriate, documented using an inline user manual including representative samples.

The user gets well introduced to the important concepts of the Service Cutter. These concepts include:

\begin{enumerate}
	\item Nanoentities
	\item Coupling criteria, their meaning and decomposition impact
	\item User representations and their impact on coupling criteria
	\item Coupling criteria priorities 
\end{enumerate}

Often used configuration like coupling criteria priorities should be shown directly to the user to encourage its usage. More advanced configuration like the values of characteristics should be accessible for the user but do not need to be shown directly in a standard workflow. 

\subsection{Simplicity}

A simple system analysis can be achieved with not more than 5 clicks. All steps are provided with useful defaults that can be changed.

\subsection{Performance}

All regular user interactions should not take more than one second.

Calculations of service cuts should meet the following conditions assuming a data set of 2000 nanoentities.

\begin{itemize}
\item Calculations that are used once per day should take less than 10 minutes.
\item Calculations that are used once per minute should take less than 5 seconds.
\end{itemize}

\subsection{Monitoring, Logging, Deployment, Availability}

The scope of this thesis is to prototypically implement the Service Cutter. Operational aspects only need to be covered on a basic level:

\begin{itemize}
	\item Log files should be written using SLF4J\cite{slf4j}.
	\item Deployment should be provided with Docker\cite{docker} containers.
\end{itemize} 

\subsection{Fault Tolerance}

As the prototype does not need to handle every possible use case or unexpected input, a common error handling needs to be built into the application and ensures that operations continue even in case of unexpected input or state.

\subsection{Maintainability}

In case of a success the prototype might be the basis for further development or other thesis projects. It should be built in clearly separated modules (or even remote services) to ensure good maintainability. At least the following modules need to be clearly separated:


\begin{enumerate}
	\item Internal representation of coupling criteria
	\item Data of a users system (nanoentities, use case definition etc.) 
	\item Decomposition algorithm (Solver)
	\item User interface
\end{enumerate}

If one or multiple of the modules are implemented as physical services the remote \gls{API} needs to be implemented as RESTful HTTP interface. 

The application should leverage existing open source frameworks or libraries wherever possible to ensure a minimal maintenance effort.

\subsection{State-of-the-Art Technology}

To support further development of the Service Cutter the prototype needs to be implemented in state-of-the-art technology. Our industry partner has suggested the following technology stack:

\begin{enumerate}
	\item Java Spring\cite{spring} as a base framework. 
	\item JHipster\cite{jhipster} with Spring Boot\cite{springboot} for project setup.
	\item AngularJS\cite{angularjs} and Bootstrap\cite{bootstrap} for the user interface.
	\item Docker\cite{docker} for container and deployment configuration and handling. 
\end{enumerate}

\subsection{License}

All involved parties decided to release the Service Cutter under the terms of the Apache 2.0 open source license.