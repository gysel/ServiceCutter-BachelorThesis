\documentclass[hsr-ba,english]{hgbthesis}
% Zulässige Class Options: 
%   Typ der Arbeit: diplom, master (default), bachelor, praktikum 
%   Hauptsprache: german (default), english
%%------------------------------------------------------------
\errorcontextlines 10000
% !TeX spellcheck = en_US

\graphicspath{{images/}}    % wo liegen die Bilder? 
\bibliography{literatur}  	% Angabe der BibTeX-Datei, % utf8-change

\usepackage{multirow,slashbox,courier,lscape,tablefootnote,hyperref}
\usepackage{longtable}
\usepackage{wrapfig}
\usepackage[toc]{glossaries}

% code listing packages
\usepackage{geometry}
\usepackage{listings}
\usepackage{color}
\usepackage[usenames,dvipsnames,svgnames,table]{xcolor}

\usepackage{pdfpages}
\usepackage{framed}

\usepackage[all]{nowidow}

\usepackage{graphicx}

\usepackage{epigraph}
\usepackage{csquotes} % \enquote{quoted text}

\setacronymstyle{long-short} % creates example(EX) for the first usage and EX for the following usages of an acronym
\makeglossaries


\loadglsentries{glossary}

% no hyphenation for the following words:
%\hyphenation{example}


%%%----------------------------------------------------------
\begin{document}
%%%----------------------------------------------------------

% use \bigskip for paragraphs that to not belowng to the section anymore. 
% (e.g. conclusions of chapters)
\bigskipamount=30pt

% Einträge für ALLE Arbeiten: --------------------------------
\title{Service Cutter}
\subtitle{a catchy subtitle...}
\author{Michael Gysel \& Lukas K\"{o}lbener}
\studiengang{Computer Science}
\studienort{Rapperswil}
\abgabedatum{2015}{12}{18}	% {YYYY}{MM}{DD}

%%% zusätzlich für eine Bachelorarbeit: ---------------------
%\semester{Spring semester 2015} 
%\gegenstand{Enterprise Application Integration} 
%\betreuer{Olaf Zimmermann} % oder \betreuerin{..}

%\strictlicense  % erzeugt restriktive Lizenzformel

%%%----------------------------------------------------------
\frontmatter
\maketitle
\setcounter{tocdepth}{1}
\tableofcontents
%%%----------------------------------------------------------

\chapter{Abstract}

Decomposing a software system into smaller parts has been an important discipline in our industry for many decades. With the rise of distributed systems, it has become more important to split a system into low coupled and high cohesive parts. Architectural styles like Service Oriented Architecture (SOA) and currently Microservices tackle the many challenges of such systems but remain vague on the art of how to decompose a system into services.

With the help of our industry partner Zühlke, our supervisor Prof. Dr. Olaf Zimmermann, and existing literature, we introduce a structured way to service decomposition by providing a comprehensive coupling criteria catalog.

We embodied these coupling criteria in the Service Cutter, a prototype that extracts coupling information out of well-known concepts like domain models and use cases. Using this information, the Service Cutter’s mission is to produce service cuts to assist an architect’s decomposition decisions. 

A scoring system is defined to automatically interpret coupling data. By employing a weighted, undirected graph and clustering algorithms the Service Cutter produces service cuts that minimize coupling between services while ensuring high cohesion within a service. 

Tests with two sample projects not only met our expectations but produced reasonable service cuts that have not been considered before. 

With a more structured way to service decomposition, the Service Cutter demonstrate that automated decision assistance is a promising way. The thesis lays the foundation for further projects in this area. 
			

%%%----------------------------------------------------------
\mainmatter         % Hauptteil (ab hier arab. Seitenzahlen)
%%%----------------------------------------------------------

% Don't show "Chapter X" titles for every Chapter
\makeatletter
\renewcommand{\@makechapterhead}[1]{%
\vspace*{50 pt}%
{\setlength{\parindent}{0pt} \raggedright \normalfont
\bfseries\Huge\thechapter.\ #1
\par\nobreak\vspace{40 pt}}}
\makeatother


\chapter{Management Summary}


\textbf{Context}

A big challenge of writing software has always been to keep the created source code manageable and maintainable. Early in the history of software development, modules have been used structure the code in manageable pieces and make it reusable. With the rise of networks and distributes systems, remote interfaces became popular and engineers started to implement services offering a remote interface to other components.

Several methodologies exist to guide a software architect when he designs services. \enquote{Service Oriented Architecture} is especially common in enterprise environments, the style of building microservices became popular in recent years. Leaving technical differences aside, both approaches share a common challenge: How can a big collection of data and functionality be decomposed into smaller pieces while retaining high cohesion and low coupling.

\textbf{A Structured Approach to Decoupling}

When talking about cohesion and coupling we found that there is no extensive description in existing literature defining the actual factors thereof in distributed software systems. We therefore compiled a catalog of 14 coupling criteria that aims to form a comprehensive but not conclusive collection. 

Those coupling criteria help a software architect to structure internal and external dependencies. 

%TODO sollen wir die einzelnen kriterien hier auflisten?

\begin{figure}[H]
	\includegraphics[scale=0.4]{diagrams/CouplingCatalog.pdf}
	\caption{Coupling criteria catalog}
	\label{fig:cc-catalog-mgmt-summary}
\end{figure}

Complementary to the coupling we also described an approach to quantify the coupling in a software system. We set nanoentities in relation to each other and assign a score to each tuple where a coupling exists. A score is a number between -10 and 10 given for a specific coupling criteria representing the strength of the coupling.

The exact importance of coupling is highly dependent on the context of a software system. Security and consistency for example are significantly divergent in a banking environment compared to an online social network. To reflect this we rate the coupling criteria using priorities before we sum it up to a final score.

All scores of coupling between nanoentities are collected and utilized to construct a weighted undirected graph. The vertices represent the nanoentities and the weighted edges embody the strength of the coupling between two nanoentities.

We implemented a tool called the \enquote{Service Cutter} as of a proof of concept of the scoring and decomposition approach.

\begin{figure}[H]
	\includegraphics[scale=0.65]{images/ServiceCutter.png}
	\caption{Screenshot Service Cutter}
	\label{fig:ServiceCutter-mgmt-summary}
\end{figure}

\textbf{Suggested Service Cuts}

Once the graph is constructed, a graph clustering algorithm is used to calculate clusters so that as few edges as possible are cut. A cut edge equals with the coupling between the created services. This process produces candidate service cuts with high cohesion and low coupling.

We utilized two different algorithms to calculate the clusters. Girvan-Newman is deterministic and requires the desired number of clusters to be specified. The \enquote{Epidemic Label Propagation} algorithm by Raghavan and Leung is non-deterministic and computes the optimal number of clusters by itself. The two algorithms seem to be complementary and provide reasonable candidate service cuts. 


\textbf{Discussion}

We performed tests based on a imaginary \enquote{Trading System}, heavily inspired by real banking software, and the DDD sample application \enquote{Cargo Tracking}. On both domain models the Service Cutter suggests meaningful service cuts.

The topic of coupling in distributed systems is very broad and leaves a lot of room for further research. Out thesis suggests that coupling in software systems is quantifiable beyond the source code level.

\textbf{Outlook}

The current Service Cutter implementation is a proof of concept based on the already comprehensive coupling criteria catalog. Further projects should be focused on the many potential features described in the Future Work Chapter as well as further improvements of the scoring process and algorithms. The Service Cutter leaves room for enhancements on both side of the tool chain: Capturing data as well as providing reusable outputs.

\chapter{Introduction}

Starting with Microservices, what news they bring (deployability) and what old, underlying problem they try to solve (decomposition)

These: There is more to service decomposition than "Belongs to bigger concept defined by noun or verb(Chris Richardson)
=> Goal/Idea

- Identify coupling criteria and constraints after which service 
decomposition should be done. 

- Build a tool that takes data fields and applied coupling criteria and constraints as input and produces suggestions on how to split the system in different parts. 


\section{Scope}

Identification \& Implementation not in scope

Focus only on data, not on logic. (??)

No focus on ways to handle coupling (async, interfaces, messaging, CAP, REST, etc.)


We should define what is part of our scope and what not. We focus only on functional partitioning (Y-axis scalability) and neglect hardware or technological impacts on service decomposition. We’re also not going into how to implement or soften service dependencies with events, RPC, REST, Messaging, abstract interfaces etc. 

TODO: is "logic" in scope of the thesis? Sometimes a bounded context might depend on a lot of data fields of another BC, but in the ends all it want's is a single processed result of this number that could be served by the other BC. How do we handle this? => Could this be modelled as a data field itself with a new kind of coupling criteria "calculated by"?

Begriffe: Module, Service, Microservice, Komponente, DDD, Bounded Context, Entity, Aggregate, Classes, Packages
=> Microservice vs. Bounded Context vs. Service definition by Udi Dahan, what is it that we produce?
=> Bounded context is not really accurate. it defines the logical borders of a system, but there might be other coupling criteria (not logical ones?) that define the broders of a "thing"

Service
Over-defined but Udi Dahan uses a good definition that suits our result. 
=> Fowler Definition 
Service Boundary Recommondations
Service/BC > Microservices 
AppArch 3 Ebenen von Services
Drei definitionen von SOMA/Services, User / Architect / Developer


Entity: Lifecycle management \& identify (=> composition?)
Local: Module, remote: Microservice



\chapter{Analysis}
\label{cha:analysis}

introduction sentence
=> Define decomposition

\section{What is a Service?}

The term \textit{service} is one of the most used words in the field of software architecture and has been differently defined in many papers, books, and blog posts in numerous ways and various contexts. 

The \enquote{4+1 View Model of Software Architecture} by Philippe Kruchten\cite{fourPlusOne} describes software architecture using multiple views like a \mbox{Logical-}, \mbox{Physical-}, \mbox{Development-} or  \mbox{Process-}View. During the research for this thesis we discovered that the difficulty to clearly define the word Service lies in the fact that different definitions focus on contrasting views. For this thesis we use multiple definitions for the word Service depending if we talk about the \textit{logical} or the \textit{technical} view of a service.

\subsection{Logical Service}

\begin{quotation}
A service is the technical authority for a specific business capability --- Udi Dahan\cite{serviceDefinitionDahan}
\end{quotation}
   
This definition focuses more on the logical or business view of a service than its technical representation. All data and rules required to fulfill a business specification are owned by one and only one Service. 

Udi Dahan implies that a Service is not restricted to a specific application, process, technology or layer. In fact it contains required layers itself, including databases, logic, and \gls{UI} code.

A logical service is autonomous and composed from many processes, webservices or databases, but keeps a clear boundary and interface against the outer world. Communication with other parts of the system only happens on a well defined interface on a common communication channel.

\subsubsection{Bounded Context}

Another concept describing logical Services is the bounded context as defined in the Domain-Driven Design:\cite{evans2014domain}

\begin{quotation}
	A description of a boundary (typically a subsystem, or the work of a particular team) within which a particular model is defined and applicable.
\end{quotation}

A model only used within one bounded context is defined and visible only in that context. Accordingly a model used in multiple Services needs to have a globally shared definition, defined as \textit{Published Language} in the context of \gls{DDD}\cite{evans2014domain}:

\begin{quotation}
	The translation between the models of two bounded contexts requires a common Language.
\end{quotation}

The process of Service decomposition as done by the Service Cutter automatically defines the published language of the system. 

\subsection{Technical Service}

Martin Fowler describes a Service as following:

\begin{quotation}
	A service will be used remotely through some remote interface, either synchronous or asynchronous.\cite{fowlerIoC}
\end{quotation}

This definition by Martin Fowler is much more technically oriented and is close to what recently has been advertised as a \textit{Microservice}:

\begin{quotation}
	In short, the microservice architectural style is an approach to developing a single application as a suite of small services, each running in its own process and communicating with lightweight mechanisms, often an HTTP resource API.\cite{fowlerMicroservice}
\end{quotation}


A process providing a remote API might provide business logic, pure technical functionality or a data store, of which the latest is often seen nowadays wrapped by a RESTful HTTP API. A technical Service might be congruent with a logical Service but very often more complex cases split logical Services in multiple technical Services. 

\subsection{Should the Service Cutter produce Logical or Technical Service Candidates?}

While every logical reason to define Service boundaries applies to logical and technical Services, technical reasons like technology constraints might define additional technical Service boundaries. We do not strictly define what kind of Services the Service Cutter suggests. But given that the Service Cutter mostly focuses on logical criteria it could well be that a suggested Service needs to be further decomposed due to technological reasons.

The factors driving the decomposition of a system into Services are discussed in more detail in the next section.

\section{Driving Factors for Service Decomposition}

Well experienced software architects decompose systems by reason of multiple factors to ensure a maintainable, stable and secure system with business relevance and good performance. This section describes the factors mostly considered by architects. The content of this section is based on research, our own experience and discussions with our industry partner Zühlke Technology Group AG. 

\subsection{Decomposition within the Service Cutter}

The Service Cutter is based on data fields. Therefore Service decomposition for the Service Cutter is simply the act of grouping fields into containers. If a field is assigned to a Service, the Service becomes the data owner and is the only instance allowed to create, update or delete instances of the field.

%TODO: is "logic" in scope of the thesis? Sometimes a bounded context might depend on a lot of data fields of another BC, but in the ends all it want's is a single processed result of this number that could be served by the other BC. How do we handle this? => Could this be modelled as a data field itself with a new kind of coupling criteria "calculated by"?


\subsection{Decomposition Principles}

Decomposition --- the craft of splitting a system into smaller parts --- has been a main discipline for programmers since early in the history of our industry. David L. Parnas published a paper entitled \enquote{On the Criteria To Be Used in Decomposing Systems into Modules} in 1972\cite{parnaDecomposing}. Shortly after, the terms \textit{coupling} 
and \textit{cohesion} as software design metrics appeared as part of the Structured Design technique\cite{structuredDesign}:

\begin{description}
	\item[Loose Coupling] In computing and systems design a loosely coupled system is one in which each of its components has, or makes use of, little or no knowledge of the definitions of other separate components.\cite{looseCoupling}
	\item[High Cohesion] In object-oriented programming, if the methods that serve a class tend to be similar in many aspects, then the class is said to have high cohesion. In a highly cohesive system, code readability and reusability is increased, while complexity is kept manageable.\cite{highCohesion}
\end{description}

Robert Martin later described a general principle to achieve loose coupling and high cohesion:

\begin{description}
	\item[Single Responsibility Principle] Gather together the things that change for the same reasons. Separate those things that change for different reasons.\cite{SRP}
\end{description}

Starting from these principles, we analyzed different types and reasons for coupling and cohesion as described in the next section.

\subsection{Coupling Criteria}
\label{sec:couplingCriteria}

A Coupling Criteria describes a reason why two data fields should or should not be owned by the same Service. These Criteria define the semantic model on which the Service Cutter is built on. 


TODO: Coupling Criteria Catalog


\subsection{What is a good Decomposition Solution?}
\label{sec:decompositionRequirements}

Based on the Coupling Criteria described above, we define a good decomposition solution as following:

\begin{enumerate}
	\item Not violating any constraints.
	\item Putting as few data fields together which have different characteristics as possible.
	\item Each Service should depend on as few data of other Services as possible (A use case should cross as few Service boundaries as possible).
	\item The amount of data fields a Service depends on should be similar for all Services.
	%TODO: This is a cite by Udi Dahan internal workshop material, how can we cite this?
	\item Not too many Services which leads to the nanoservice antipattern\cite{nanoservice}.
	\item Not too few Services, which leads to a monolithic architecture.
\end{enumerate}

\section{Existing Decomposition Solutions}

TODO: Market analysis. Java Code analysis, GraphGist project to analyse service dependencies etc...




%TODO User stories as most important criteria: check Ivar Jacobson, oose, a use case driven approach

%TODO: Should we describe non relevant coupling criteria as well? (Hiding design decisions etc, see CC catalog doc)

%TODO: (Bring result \& discussion already here?)



\chapter{Requirements}
\label{cha:requirements}

%TODO: describe the service cutter as a solution finding tool (not a architecture builder) either here or in introduction

This chapter describes the (non-)functional requirements of the target solution and covers the characteristics of it's target users. 

The requirements in this chapter are not prioritized. As part of the sprint planning meetings, these requirements are transformed to tasks and prioritized. The description in this chapter is meant to provide a high level overview and establish a common sense between all stakeholders of this project.


\section{Personas}

The personas are inspired by a series of discussions in meetings and workshops with our stakeholders.

All users of the system are experienced software architects or developers interested in the educational aspects of the tool.

\subsection{Junior Jedi-Master}

Junior has been a fast learning and aspiring developer since he graduated from university with a master's degree in information technology eight years ago. In his new job, Junior finds himself in the role of an architect for a new and promising product that enjoys the support of well known investors. In consequence of his experience in distributed systems, he has been assigned with the task to decompose the system's business model into logical services of which each will be split into multiple separately deployable microservices. 

Junior strongly believes in automation and using every tool available to support and complete his work. For his current project he plans to try the Service Cutter as a foundation and verification of his architectural decisions. 


\subsection{Walter Wisenheimer}

Walter is an architect with many years of experience in the industry and has built numerous systems already. Walter has seen many automation concepts and tools failing their goals. In Walter's view a natural and obvious outcome -- an architect's world is too complex to model and automate in a system or algorithm. 

After a longer discussion with an very motivated junior developer who recently joined his company, Walter starts to see the benefits of the well structured format the Service Cutter organizes architecturally relevant information. Using the tool to structure and document his systems characteristics might be of benefit as currently a lot of his precious knowledge remains tacit\cite{zimmermann2015architectural}. 

Walter decided to try the Service Cutter to structure the information for the current project he is working on.


\subsection{Stan Student}

Stan studies Computer Science and as part of his class in Service oriented Software Architecture he is supposed to design a set of services for the Cargo Tracking\cite{dddGithub} domain. The Service Cutter guides him through the important decisions by asking a set of questions and presents him a set of possible service cuts. 

Being overwhelmed of all the data requested by the Service Cutter he would like to configure the tool to only focus on the data he got provided in his exercises. 

Stan then discusses the advantages and disadvantages of the presented options with his fellow students. He furthermore asks his Professor about the to him unknown criteria the Service Cutter requested and why that information might have an impact on service orientation. 


\subsection{Tom Tutor}

Tom wants to introduce his students to the software architecture craft. He uses the Service Cutter to visualize the different ways of distribute data into services during his lectures. By changing the calculation parameters he can demonstrate that software architecture mostly depends on the context of the requirement. The same problem might have different solutions in varying circumstances.

TODO: add enterprise architect role here?
%TODO: enterprise architect role?


\section{Functional Requirements}

The Service Cutter faces the functional requirements presented in this
section.

%TODO low prio: Show warnings as part of the output should a predefined service have a significant negative effect!

\subsection{Coupling Criteria}

All \textit{Cohesiveness}, \textit{Compatibiltiy} and \textit{Constraints} criteria must be supported. The Service Cutter needs to rate for each criteria which nanoentities should be placed in one service and which need to be separated over multiple services. Within this rating, every coupling criterion is equally respected.

Criteria of type \textit{Communication} do not describe the need to merge or separate nanoentities but characterize nanoentities suitable for inter service communication and are handled with a lower priority than the other criteria. 

\subsection{User Representations}

To achieve better usability, the user does not need to now the exact definitions of the coupling criteria or their internal structure. He can use well known concepts called \textit{user representations} to describe his system, from which the relevant nanoentities and coupling criteria data will be extracted by the Service Cutter. At least the following User Representation must be supported by the system:

\begin{description}
	\item[Entity Relationship Diagram] containing \glspl{entity} and their relations to each other. Each entity contains a list of nanoentities building the basis for a systems analysis.
	\item[Use Case] containing at least information about all nanoentities read or written within a particular case. 
	\item[Categorization] of the nanoentities in different characteristics per compatibility coupling criteria. 
\end{description}

\subsection{Priorities}

As every coupling criteria is respected equally in the rating process, the user needs an additional way to prioritize criteria according to its system characteristics. For an application processing financial data, security might be more important than network traffic. For an application that needs to support high volumes of data, volatility and resilience might be a primary focus.

The system should allow to change priorities for all supported coupling criteria while providing reasonable default values. 

\subsection{Candidate Service Cuts}

Based on the input provided by user representations and the defined criteria priorities, the Service Cutter produces a set of candidate service cuts for the analyzed system. The requirements for such are documented in Section \ref{sec:decompositionRequirements}

\subsection{Visualize Published Language}

If relevant user input (e.g. use case definitions) is available, the Service Cutter is able to tell which service depends on which nanoentities of other services. These dependencies need to be visualized. All nanoentities shared between different services make up the published language of the system and need to be visualized as such.

\section{Nonfunctional Requirements}

The following non-functional requirements should be satisfied by the Service Cutter.

\subsection{Deployment}

No software installation is required. The latest version of any popular internet browser (Chrome, Firefox, Internet Explorer) is sufficient.

On the server side Java is used to implement the business logic. An appropriate storage and web server technology will be evaluated as part of the design phase. The deployment of the different components is managed using \gls{docker} containers.

\subsection{Usability}
\label{sec:usability}

A software architect should be able to use the software without any training. All controls are clearly named and, where appropriate, documented using an inline user manual.

Up to 500 data fields and 50 entities are manageable without losing track.

The base layout is responsive and adapts to smaller screens such as smartphones. However the tool is mostly used on devices such as laptops and the controls are therefore optimized for use on screens that are at least 15 inches wide and used with a mouse and a keyboard.

\subsection{Simplicity}

A simple workflow can be achieved with a few clicks. All steps are provided with useful defaults that can be changed.

\subsection{Information Security}

The web application is secured using an authentication and authorization implementation. Any other internal components such as the database or web services are hidden behind the servers firewall and therefore do not need any special security measures.

The uploaded data models are initially shared amongst all registered users.

\subsection{Separation of Concerns}

The software is structured in a way that separates the application into several concerns running in separated containers that are managed individually.

\subsection{Performance}

All regular user interactions should not take more than one second. The service boundary calculation may take up to 10 seconds depending on the data model size. The benchmark is the upper limit outlined in section \ref{sec:usability}.

\subsection{Monitoring}

The software is not business critical and no explicit monitoring is required.

\subsection{Availability and Fault Tolerance}

The software does not implement any high availability measures. A common error handling is built into the application and ensures that operations continue even in case of unexpected input or state.

\subsection{Maintainability}

The application should leverage existing open source frameworks or libraries whereever possible to ensure a minimal maintenance effort.

\subsection{Logs}

Log files should be written using SLF4J\cite{slf4j} provided by Spring Boot\cite{springboot}.

\subsection{License}

No GPL? to be discussed %TODO


\bigskip

After defining all (non-)functional requirements, the next Chapter outlines design and implementation steps which were taken to satisfy these requirements.




\chapter{Design and Implementation}
\label{cha:implementation}

\section{Decomposition Algorithm}

\subsection{Approach \#1: Clustering of a Weighted Graph}

\subsection{Approach \#2: Set Rating}

\subsection{Approach \#3: Constructing Services - a Heuristic Approach}

\section{Prototype} 

\subsection{Design}

\subsection{Technology}

\subsection{Infrastructure}

\subsection{Information Security}

The web application is secured using an authentication and authorization implementation. Any other internal components such as the database or web services are hidden behind the servers firewall and therefore do not need any special security measures.

The uploaded data models are initially shared amongst all registered users.

\subsection{User Interface}

The base layout is responsive and adapts to smaller screens such as smartphones. However the tool is mostly used on devices such as laptops and the controls are therefore optimized for use on screens that are at least 15 inches wide and used with a mouse and a keyboard.


\bigskip
After covering the important design and implementation aspects, the next chapter assesses the built solution described in this chapter against the defined requirements.


\chapter{Working with the Service Cutter}
%TODO: Does this chapter make sense? it could cover the process of working with the service cutter, finding the needed information, intepreting the results and using it for decision finding and documentations (ADs). Giersche highly recommended/requested this. 

How do you initially populate all the coupling data? LOTS of knowledge is required!
o   How do you organize such an interview?

\chapter{Conclusion}


\section{Future Work}

* Caching (Data Redundancy) as a way to reduce temporary reduce coupling between services could be a further suggestions of the Service Cutter

* Suggestions for APIs (REST) or Message types 

* Create a Questionnary similar to JHipster to find out the importance of criteria for a specfic system. What questions do we need to ask to find out the characterstics of a system? 


%%%----------------------------------------------------------
%%%Anhang
\appendix



\chapter{Parking Lot}
document not so important findings here	

* Resilience not only per Field definition but also per Instance definition 

* Find Aggregates from Consistency data? Aggregates should not be split between services.

* Decomposition mode: logical or phyiscal. Or: Microservice vs. Module?

* Idea: Only suggested logical services (similarity), then zoom in to see technical services (distance). Algorithm could be splitted into two parts?


\chapter{Project Management}
\label{cha:projectmgmt}

This Appendix documents aspects of project management like methodology, roles, environment, quality management, and risk management. The project was time boxed and started on \formatdate{14}{9}{2015} and ended on \formatdate{18}{12}{2015} which implies 14 working weeks. 

This section documents our milestones and sprint plan.

\subsection{Milestones}

Apart from the appointed project end on the \formatdate{18}{12}{2015}, the following milestones ensure a timely progress.

\begin{enumerate}
\item \textbf{\formatdate{23}{10}{2015}} - Finished chapters \enquote{Context}, \enquote{Idea} and \enquote{Requirements}
\item \textbf{\formatdate{20}{11}{2015}} - Finished chapters \enquote{Solution / Architecture / Design}
\item \textbf{\formatdate{14}{12}{2015}} - Finished chapters \enquote{Result}, \enquote{Discussion} and \enquote{Future Work}
\end{enumerate}

All milestones relate to the documentation. The implementation tasks are scheduled to be mostly finished with the milestone 2 as well. A more detailed breakdown is presented in Appendix \ref{sec:projplan}.
\section{Sprint Plan}
\label{sec:projplan}

The project is organized in 4 regular sprints with each two weeks length. 

\begin{itemize}
\item 1st Sprint: \formatdate{14}{09}{2015} - \formatdate{25}{09}{2015}
\subitem Refinement of thesis concept
\subitem Implementation of prototype
\item 2nd Sprint: \formatdate{12}{10}{2015} - \formatdate{23}{10}{2015}
\subitem Development of algorithm
\subitem Development of Coupling Criteria catalog
\item 3rd Sprint: \formatdate{09}{11}{2015} - \formatdate{20}{11}{2015}
\subitem Finish functional requirements
\subitem Test significantly large project
\item 4th Sprint: \formatdate{30}{11}{2015} - \formatdate{11}{12}{2015}
\subitem Refine scoring system, fine tune sample models
\item 5th Sprint: \formatdate{14}{12}{2015} - \formatdate{18}{12}{2015} (one week)
\end{itemize}

The 5th sprint will solely be used to finish the documentation.

%TODO müssen wir beschreiben, was wir zwischen den sprints machen?
\section{Risk Management}

%TODO: we should have a measurement for impact and "Eintretungswahrscheinlichkeit" to provide a proper risk mgmt analysis.

To assess the risk associated with this project, a comprehensive list of possible risks and their mitigations is described in the following Section. 

The risk assessment shown in Table \ref{tab:toprisks} is based on a literature survey\cite{arnuphaptrairong2011top} taken in 2011. Two custom lists of project specific risks are documented in Table \ref{tab:projmgmtrisks} and Table \ref{tab:projtechnicalrisks}.

\begin{table}[H]
\begin{tabular}{|c|p{120pt} p{100pt} p{140pt}|}
\hline \# & Risk & Impact & Evaluation \& Mitigation \\ 
%\hline 1 & Stakeholders have opposing requirements. & It is impossible to prioritise requirements. & Discuss implementation approaches and priorities in meetings with all stakeholders present. \\ 
%2 & Wrong priorities are defined and unimportant features implemented first. & Important features are left out. & Validate priorities and functionality with all stakeholders on a regular basis. \\ 
1 & The criteria on which architects create service cuts are too complex or to context specific to be modeled in a system. & The idea of an automatic Service Cutter will not be possible to realize. & Analyze coupling criteria early on in the project and review with our industry partner. Focus more on the conceptual part if modeling the criteria proves to be impossible. \\
2 & The scope of the thesis is too wide and the domain area too complex to be covered within the time box available. & No significant result can be produced within the time available. & Focus first on the conceptual part of identifying coupling criteria, which is already a significant result. Then use an iterative approach to maximize business value at all time, focusing on a proof of concept of the decomposition algorithm. \\
3 & As the thesis includes many different tasks and facets, loosing too much time in details is likely to happen. & The main goal cannot be reached because resources are not available. & Review work done recently with our advisor and the industry partner to ensure common priorities. \\
4 & A team member faces health issues and cannot continue to work on the project. & The project scope is impossible to fulfill. & The project scope has to be renegotiated with our advisor and the industry partner. \\ 
5 & The idea and implementation of a Service Cutter works, but the tool is not accepted by its target users as the usability is insufficient or the effort to provide the needed data is too high. & Service Cutter will not be used by its target group. & Define clear non functional requirements for usability and simplicity. Try to find well known concepts (user representation) for the input data. \\
\hline
\end{tabular}
\caption{Project-Specific Management Risks}
\label{tab:projmgmtrisks}
\end{table}

\begin{table}[H]
\begin{tabular}{|c|p{80pt} p{140pt} p{140pt}|}
	\hline \# & Risk & Impact & Evaluation \& Mitigation \\ 
	\hline 1 & Calculating service cuts using a clustering algorithm does not produce usable results. & An important functional part of the service cutter cannot be implemented. & Also evaluate other types of algorithms and functional alternatives. Discuss ideas early on with experts available at \gls{HSR} to validate feasibility. \\ 
	2 & Unstable development environment & Analyzing and fixing problems takes too much time and causes a delay in the project plan. & Make use of established development tools and agree on a common version of Java, the development environment and plugins. \\
%	3 & Architectural decisions introduce unnecessary complexity. & Implementation of functionality is  & Assess every architectural decision regarding its impact towards the complexity. Where possible prefer established technologies. \\ 
	3 & Developed source code is not covered with unit tests. & Future refactorings are hard to perform without appropriate test coverage. Unit tests serve as a functional documentation as they specify the intended behavior of a piece of code. & All written software should be covered with automated unit or integration tests. The code coverage plugin of Jenkins helps to monitor the test coverage. \\
	4 & The suggested technology stack by the industry partner is difficult to install or requires significant training investments. & The prototype and user interface will require too much time to be build so that the conceptual analysis and algorithm development will not get enough resources. & Evaluate and build a first prototype within the first sprint to have enough time to discuss impact of technology constraints early.\\
	5 & Project infrastructure outage & JIRA, GitHub or the project server goes down and the progress is therefore delayed.  & All components are supported by a company and professional support should therefore be available quickly.  \\ 
	6 & Personal infrastructure failure & A personal laptop of one of the team members stops working. & HSR desktops are available and can be used as replacement hardware. Furthermore an personal spare notebook owned by Michael Gysel is available as well.  \\ 

	\hline
\end{tabular}
\caption{Project-Specific Technical Risks}
\label{tab:projtechnicalrisks}
\end{table}

\begin{table}[H]
\begin{tabular}{|c|p{80pt} p{140pt} p{140pt}|}
\hline \# & Risk & Impact & Evaluation \& Mitigation \\ 
\hline 1 & Misunderstanding of requirements & The final product consists of features that do not comply with the requirements of the stakeholders. & Likely to happen as the industry partner's resources are limited for requirement and deliverable reviews. At least one meeting per sprint needs to be held to ensure the work is in line with our industry partner's expectations.  \\ 
2 & Lack of management commitment and support & The project lacks funding or resourcing. & The project sponsor and other stakeholders committed to attend status and review meetings whenever allowed by their schedules. \\ 
3 & Lack of adequate user involvement & Requirements and solutions cannot be validated with users. & An extra workshop apart of the sprint status meeting will be held to review the conceptual work on coupling criteria. Acceptance tests with potential external customers would help to improve user feedback. \\ 
4 & Failure to gain user commitment & End users may not be able to provide required progress reviews or required contributions towards the requirement specification. & As the thesis goal is a prototype as a proof of concept, end user involvement is not in scope. \\ 
5 & Failure to manage end user expectation & End users may not be able to use the product or refuse to do so. & As the thesis goal is a prototype as a proof of concept, end user involvement is not in scope. \\ 
6 & Changes to requirements & Already implemented functionality turns out to be unnecessary or based on wrong assumptions. & Likely to happen as the stakeholders have varying priorities and conceptions of the product to be developed. Has to be mitigated by reviewing requirements as part of the regular meetings. Absent stakeholders need to be informed of all decisions taken at such meetings. \\ 
7 & Lack of an effective project management methodology & Project is delayed and generated business value is impacted. & The project will be managed using Scrum, a methodology that is already known to all involved parties. \\ 
\hline 
\end{tabular} 
\caption{Top Software Risks Evaluation}
\label{tab:toprisks}
\end{table}

\subsection{Risk Assessment}

As all mitigation measurements described in the last Section were implemented and therefore none of the described risks put the project seriously at risk.

%TODO review and update

% All taken measures and used methodologies proved to be good choices to ensure good
% project management. The project did not suffer from management overhead nor were
% important aspect left out or forgotten during the project
\section{Development Environment}

Figure \ref{fig:devenvironment} outlines all components of the development environment. The Service Cutter is developed with Java 8 using Eclipse Mars~\cite{eclipsemars}. \gls{HSR} has provided a \gls{VM} on which the \gls{docker} images are running. On the same server a \gls{CI} Jenkins server pulls for changes from the Git repositories on GitHub to build and test the different projects. The Docker images are built using a Maven plugin on Jenkins and then pushed into the local Docker repository. Table \ref{tab:vm} indicates the software installed by us on the server.


\begin{table}[H]
\begin{center}
\begin{tabular}
{|p{100pt} p{80pt}|}
\hline \textbf{Software} & \textbf{Version} \\ 
\hline Java & 1.8.0\_60 \\ 
\hline Jenkins & 1.629 \\ 
\hline Docker & 1.8.2 \\ 
\hline Docker Compose & 1.4.1 \\
\hline Apache Maven & 3.0.5 \\
\hline Node.js & 0.10.25 \\
\hline NPM & 1.3.10 \\
\hline 
\end{tabular} 
\caption{Installed software on sinv-56064.edu.hsr.ch (152.96.56.64).}
\label{tab:vm}
\end{center}
\end{table}

\begin{figure}[H]
	\centering{\includegraphics[scale=0.7]{diagrams/DevelopmentEnvironment.pdf}}
	\caption{Development Environment}
	\label{fig:devenvironment}
\end{figure}


\chapter{Meetings}

All minutes of meetings of the status meetings and workshops are listed in
this chapter.

\section{Project Status}

Every sprint one or two status meetings where held to track the progress.

\subsection{Kick Off \formatdate{15}{9}{2015}}

\textbf{Time:} 14:00 – 15:15

\textbf{Next Meeting:} \formatdate{21}{9}{2015} 10:30

\textbf{Attendees:} Olaf Zimmermann (ZIO), Lukas Kölbener (KOE), Michael Gysel (GYS)

\textbf{Agenda}
\begin{itemize}
\item Meetings, Sprints
\item Aufgabenstellung
\item Discussion on the subject of the thesis
\end{itemize}

\textbf{Notes}
\begin{itemize}
\item The automatic REST interface of Apache ISIS could be an inspiration
\item Vaughn Vernon provides his DDD samples on \href{https://github.com/VaughnVernon/IDDD_Samples}{GitHub}
\item The provided article by Subbu Allamaraju is a must-read.
\item General approach: Start with business requirements (Use Cases, User Stories) and transform them into Components (e.g. UML). These Components are then modeled into services.
\item UML Components approach: Identification, Specification, Realization. (Our tool would probably support the step from spec to realization)
\item Possible quality attributes: Coupling, Cohesion, Security, Performance, Data volatility, Frequency of updates, Monitoring, Reconciliation, Consistency, Data invariants, Data Volumes
\item Structurizer by Simon Brown is a possible input format
\item C4: Context, Container, Components, Classes
\item Swagger is a possible output of the tool
\item Idea: Quality attributes should be specified on the level of fields instead of entities.
\item New item in the reading list: \url{https://msdn.microsoft.com/en-us/library/ms954587.aspx} 
\item NFR: Number of entities/fields to be supported
\end{itemize}

\textbf{Decisions}
\begin{itemize}
\item MoM and Meetings should be handled the same way as in the SA.
\item Project management should be < 10\% of the overall effort.
\item Model reconciliation is out of scope!
\end{itemize}

\textbf{Tasks}
\begin{itemize}
\item GYS: Send invite for the next meeting
\item GYS/KOE: Suggest dates for the intermediate presentation (end Oct, beginning of Nov)
\item ZIO: Send Aufgabenstellung via mail to GYS/KOE
\item ZIO: Bring UML Components book to the next meeting
\item ZIO: Book a room for the next meeting (21.9. 16:30)
\end{itemize}

\subsection{Status Meeting \formatdate{21}{9}{2015}}

\textbf{Time:} 10:30 – 12:00
 
\textbf{Next Meeting:} \formatdate{19}{10}{2015} 14:00-16:00 (Zühlke Office)
 
\textbf{Attendees:} Olaf Zimmermann (ZIO), Lukas Kölbener (KOE), Michael Gysel (GYS), Wolfgang Giersche
 
\textbf{Agenda}
\begin{itemize}
\item Scope and content of the thesis 
\item Project Organization / Review Meetings
\item Intellectual property rights
\end{itemize}

\textbf{Notes}
\begin{itemize}
\item Coupling Criterion / architectural significant requirements
\subitem Data confidentiality
\subitem SPI, Sensitive Personal Information
\subitem Structural volatility
\item Critique on automated scoring systems for IT architecture:
\subitem Pseudo accuracy, for example when mapping non functional requirements
\subitem Only one of many relevant criteria considered
\subitem => Make a sensitivity analysis to analyze the impact of weight-parameters
\item Define exactly what is meant with entity, bounded context and aggregates are within the system
\item Check Spark framework for algorithms
\item Check triple graph grammar for model transformations
\item Possible modes of system: Suggested bounded contexts are local (Modules, Components) or separated by network (Remote/ Microservices)
\item The data used in multiple bounded contexts should be part of the \enquote{published language} (DDD)
\item Dr. Gerald Reif will take the role of the expert
\end{itemize}

\textbf{Decisions}
\begin{itemize}
\item The target audience (Persona) are architects which use the system as assistance in taking architectural decisions
\item The produced results will be published on Github under the Apache 2.0 license
\item Meetings 
\subitem 19.10.2015, 14:00-16:00 Office Zühlke 
\subitem 22.10.2015, 09:00-10:00 HSR
\subitem 19.10.2015, 11:00-12:00 HSR
\end{itemize}

\textbf{Tasks}
\begin{itemize}
\item GYS/KOE: Review \enquote{Aufgabenstellung} and send back to ZIO
\item GYS/KOE: Send invitation for meetings
\item GYS/KOE: Authorize ZIO on Github repositories
\item Giersche: Find example project(s) with good complexity level and documentation (domain model, user stories, NFRs...). Clarify if the project can be published as an example within the thesis.
\end{itemize}

\subsection{Status Meeting \formatdate{15}{10}{2015}}

\textbf{Time:} 11:00 - 12:00

\textbf{Next Meetings:} \formatdate{19}{10}{2015} 14:00, Z\"uhlke; \formatdate{22}{10}{2015} 09:00, HSR
 
\textbf{Attendees:} Olaf Zimmermann (ZIO), Lukas Kölbener (KOE), Michael Gysel (GYS)

\textbf{Agenda}

\begin{itemize}
\item Thesis \& documentation progress
\item Outcome discussion with O. Augenstein // Complexity, alternative approaches
\item Date of intermediate presentation (Zwischenpr\"asentation)
\item Interest of industry contacts
\item Idea workshop Monday @ Z\"uhlke
\end{itemize}


\textbf{Notes}
\begin{itemize}
\item The thesis title needs to be finalized by mid November.
\item \enquote{Service} is a good candidate for the things produced by the service cutter. Conflicting definitions exist for \enquote{Service} of which some focus on business capability and others on providing a remote API. 
\item Possible example project candidates are Netstal (\url{http://wwww.netstal.com}) and the master thesis by Jonas Biedermann.
\item The clustering approach has multiple flaws which are hard to overcome. A new approach using theory of sets is inspired by the discussion with Prof. O. Augenstein and will be analyzed within this sprint. 
\item Good visualization of the service cutter input and good traceability within the tool might support the taking and documentation of architectural decisions.
\item Andreas Rinkel will take the role of the \enquote{Gegenleser} for this bachelor thesis.
\item ZIO provided further ideas and research material on the topic of coupling criteria and decomposition. 
\end{itemize}
 
\textbf{Tasks}
\begin{itemize}
\item ZIO: Send draft for legal rights agreement (15.10)
\item ALL: Sign legal rights agreement (19.10)
\item GYS/KOE: Return "UML Components" book to ZIO by next week (22.10)
\item ZIO: Find date for intermediate presentation during 9-11. November (21.10)
\end{itemize}


\subsection{Status Meeting \formatdate{22}{10}{2015}}
\label{sec:status22102015}

\textbf{Attendees}: Wolfgang Giersche, Olaf Zimmermann (ZIO), Lukas Kölbener (KOE), Michael Gysel (GYS)
 
\textbf{Next Meeting}: 19.11.2015, 11:00 @HSR / Skype

\textbf{Agenda}

\begin{itemize}
\item Admin: Sign paper regarding usage rights
\item Demo: Prototype, development environment
\item Coupling Criteria Catalog
\item Algorithms and coupling quantification
\item Sample project (not discussed)
\end{itemize}
 
\textbf{Notes}

\begin{itemize}
\item Maintainability is important. Coupling Criteria should be changeable with an appropriate effort.
\item Findings discussion on the algorithm / approach.
\subitem Calculating a score of every possible service cut is not possible.
\subitem New term: Candidate Service Cut.
\subitem Idea W.G.: Start with a heuristic approach like “Substantiv-Clustering”, a set of Candidate Service Cuts derived from a graph cluster or an analysis using only one Coupling Criteria.
\subitem Document needs to explain why we did not “just use a graph cluster”.
\item Three step approach is probably required
\subitem Determine a set of Candidate Service Cuts (perform once) (needs to be a smart solution – not a brute force approach! E.g. 100 possibilities.)
\subitem Assess all Candidates with a given set of weights to come up with a score.
\subitem Evaluate all Candidates with priorities by CC to find best choices.
\item Idea: Shake solution to improve it.
\end{itemize}
 
\textbf{Decisions}

\begin{itemize}
\item It is not an exact science – a good solution is enough.
\item No complete freeze of the CC catalog as of now.
\item Performance for 100 Entities, 2000 Fields:
\subitem Calculate Candidate Service Cuts (Steps 1\&2) – less than 10 minutes
\subitem Evaluate Candidates with parametrized weights (Step 3) – less than 5 seconds
\end{itemize}
 
\textbf{Action Items}

\begin{itemize}
\item GYS/KOE: Send documentation to ZIO for a review by the end of Oct.
\item GYS/KOE: Enhance Coupling Catalog with Decomposition Impact, Example (maybe from trading system), Measurement/Quantification, Type (Distance/Proximity/Constraint)
\end{itemize}

\subsection{HSR Status Meeting \formatdate{11}{11}{2015}}

\textbf{Attendees}: Olaf Zimmermann (ZIO), Lukas Kölbener (KOE), Michael Gysel (GYS)
 
\textbf{Next Meeting}: tbd

\textbf{Agenda}

\begin{itemize}
\item Status of thesis / development
\item Sprint Goals 3 \& 4
\item Sample Project
\item Personas
\item Final presentation
\item Discussion Thesis Review
\end{itemize}

\textbf{Notes}

\begin{itemize}
\item The cargo tracking sample application is currently being reimplemented. Maybe take a look at it to use it as test project.
\item Can we apply the service cutter to the service cutter domain model? Good for credibility to test one's own systems. 
\item Personas are very good. Add an Enterprise Architect.
\item Papers on coupling: (focus on object oriented code, not architecture!)
	\begin{itemize}
		\item \url{http://ieeexplore.ieee.org/stamp/stamp.jsp?tp=\&arnumber=1605186}
		\item \url{http://ieeexplore.ieee.org/stamp/stamp.jsp?tp=\&arnumber=4021375\&tag=1}
	\end{itemize}
\item Ideas for a renaming of \enquote{data field}:
	\begin{itemize}
		\item Micro entity
		\item Nano entity
		\item Candidate Entity
		\item Nano service
	\end{itemize}
\item Describe every concept with an example
\end{itemize}

\textbf{Additional feedback by ZIO}
\begin{itemize}
\item Explain that criteria catalog does not aim for completeness.
\item Introduce example earlier .
\item Visualize coupling types.
\item Apply Service Cutter concepts/tool to Service Cutter design (as an additional validation step in last sprint).
\item Generalize from use case/user story as input (of busienss-level operations) to conceptual components (e.g. CRC cards)?
\item Make sure that all tool output can be reproduced, or that tool output can be consumed as input (e.g. in follow on projects or subsequent but delayed iterations in a project), consider previously proposed service cuts as one special type of legacy system constraint?
\end{itemize}

\textbf{Decisions}

\begin{itemize}
\item Notation
	\begin{itemize}
	\item Web uppercase
	\item service lowercase
	\item coupling criteria lowercase
	\item Service Cutter uppercase
	\end{itemize}
\item Final term instead of \enquote{data field} should be finalized in the next meeting.
\end{itemize}
 
\textbf{Action Items}

\begin{itemize}
\item ZIO: Find date for Abschlusspräsentation. (14. – 22. Januar)
\item ZIO: Send mentioned papers/books on pages 8 and 11.
\item GYS/KOE: Prepare suggestions for meetings in December.
\end{itemize}


\section{Workshop}

One workshop was organized to refine the concept of coupling criteria.

\subsection{\formatdate{19}{10}{2015} at Z\"uhlke}

\textbf{Agenda}

\begin{enumerate}
\item Introduction
\item Brainstorming Coupling Criteria
\item Refinement of prepared Coupling catalog.
\end{enumerate}

\textbf{Attendees:} Wolfgang Giersche, Olaf Zimmermann (ZIO), Lukas Kölbener (KOE), Michael Gysel (GYS)

\textbf{Notes}

CC Cards Layout:

\begin{itemize}
\item Identity \& Lifecycle
\item Business Transactions | Volatility | Resilience
\item Security Constraints | Consistency
\item Network Traffic | Storage
\item Predefined Services | Change Management
\end{itemize}


\textbf{Decisions}

\begin{itemize}
\item 10 Coupling Criteria will be finally approved in the next meeting.
\item \enquote{Identity \& Lifecycle}, \enquote{Business Transaction}, \enquote{Volatility} and \enquote{Resilience} are the most important CC to be implemented first.
\end{itemize}

\textbf{Action Items}

\begin{itemize}
\item GYS/KOE: Finalize Coupling Criteria catalog. (22.10.2015)
\item ZIO: Provide slides about master data categorization (information integration?). (22.10.2015)
\item Giersche: Provide model (and maybe a contact) of an internal software project. (22.10.2015)
\item GYS/KOE: Prepare quantification concept for coupling criteria. (22.10.2015)
\item GYS/KOE: Add aggregation types, n-n relations and inheritance to example model. (23.10.2015)
\item ZIO: Send pictures of CC cards. (22.10.2015)
\end{itemize}

%\section{Project Management Methodology}

We chose Scrum as the project management methodology for this bachelor thesis. Scrum specifies an iterative and incremental approach which encourages a high involvement of the customer. This characteristics suit the requirements of a bachelor thesis well,
as it provides only a short project definition at project start and, in our case, requires close collaboration with our advisor and industry partner.

%
\section{Project Roles}

Scrum defines three project roles — the product owner, the scrum master and the development
team. As this bachelor thesis is an academical project, the scrum roles could
not be mapped one to one. This Section introduces the involved persons and their roles
in the project.

\subsection{HSR Supervisor}

\begin{minipage}[t]{0.25\textwidth}
	\vspace{0pt}
	\includegraphics[width=0.8\textwidth]{olz.jpg}
\end{minipage}
\begin{minipage}[t]{0.8\textwidth}
	\vspace{10pt}
	Prof. Dr. Olaf Zimmermann is the \gls{HSR} supervisor of this thesis and incorporates both, the role of the product owner and scrum master. \textit{Product} in this context refers to the thesis itself as well as the functional product. He ensures that all requirements by HSR for a bachelor thesis are met and decides in last instance about the scope of the thesis. 
	As supervisor and coach of the development team, Mr. Zimmermann also performs part of the scrum master role as he coaches the development team.
\end{minipage}


\subsection{Industry Partner}

Our industry partner Zühlke Engineering, represented by W. Giersche, ensured the functional relevancy of this thesis. Mr. Giersche contributed valuable experience from many software engineering projects. He represented part of the product owner role as he played an important role in the functional prioritization to maximize the business value. Being an experienced software architect, he also consulted our research of existing literature and contributed to the coupling criteria catalog.


\subsection{Project Team}

Michael Gysel and Lukas Kölbener formed the development team. They worked as an interdisciplinary team in which both were responsible for each part of the project. Both being Certified ScrumMasters\textregistered\cite{scrummaster}, they incorporated part of the scrum master role as they helped maintaining the product backlog and organized everything for correct sprint operation.

\begin{minipage}[t]{0.25\textwidth}
	\vspace{0pt}
	\includegraphics[width=0.8\textwidth]{lukas.jpg}
\end{minipage}
\begin{minipage}[t]{0.8\textwidth}
	\vspace{20pt}
	Lukas Kölbener is an information technology student at \gls{HSR} in his 9\textsuperscript{th} semester. He works part time as Java developer for Super Computing Systems AG in Zurich, building ticket vending machines for the public transport industry.
	\newline
\end{minipage}

\begin{minipage}[t]{0.25\textwidth}
	\vspace{0pt}
	\includegraphics[width=0.8\textwidth]{michi.jpg}
\end{minipage}
\begin{minipage}[t]{0.8\textwidth}
	\vspace{20pt}
	Michael Gysel is an information technology student at \gls{HSR} in his 9\textsuperscript{th} semester. He works part time as Java Developer for FIS, a global provider for banking and payments technologies.
	\newline
\end{minipage}



%\include{projectmgmt/risk_management}

%\include{anhang_a}	% 
%\include{anhang_b} % 
%\include{anhang_c} % 
%\include{anhang_d} % 
%\include{anhang_e} % 

%\renewcommand{\glossarypreamble}{Descriptions of Glossary items have mainly been taken from Wikipedia.org.\newline\newline}

\printglossaries


%%%----------------------------------------------------------
\MakeBibliography
%%%----------------------------------------------------------


\end{document}
