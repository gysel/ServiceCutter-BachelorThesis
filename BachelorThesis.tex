\documentclass[bachelor,english]{hgbthesis}
% Zulässige Class Options: 
%   Typ der Arbeit: diplom, master (default), bachelor, praktikum 
%   Hauptsprache: german (default), english
%%------------------------------------------------------------

% !TeX spellcheck = en_GB

\graphicspath{{images/}}    % wo liegen die Bilder? 
\bibliography{literatur}  	% Angabe der BibTeX-Datei, % utf8-change

\usepackage{multirow,slashbox,courier,lscape,tablefootnote,hyperref}
\usepackage{longtable}
\usepackage{wrapfig}
\usepackage[toc]{glossaries}

% code listing packages
\usepackage{geometry}
\usepackage{listings}
\usepackage{color}
\usepackage[usenames,dvipsnames,svgnames,table]{xcolor}

\usepackage{pdfpages}
\usepackage{framed}

\usepackage[all]{nowidow}

\usepackage{graphicx}

\usepackage{epigraph}


\setacronymstyle{long-short} % creates example(EX) for the first usage and EX for the following usages of an acronym
\makeglossaries


\loadglsentries{glossary}

% no hyphenation for the following words:
%\hyphenation{example}


%%%----------------------------------------------------------
\begin{document}
%%%----------------------------------------------------------

% use \bigskip for paragraphs that to not belowng to the section anymore. 
% (e.g. conclusions of chapters)
\bigskipamount=30pt

% Einträge für ALLE Arbeiten: --------------------------------
\title{Service Cutter (Working title)}
\author{Michael Gysel \& Lukas K\"{o}lbener}
\studiengang{Computer Science}
\studienort{Rapperswil}
\abgabedatum{2015}{12}{18}	% {YYYY}{MM}{DD}

%%% zusätzlich für eine Bachelorarbeit: ---------------------
%\semester{Spring semester 2015} 
%\gegenstand{Enterprise Application Integration} 
%\betreuer{Olaf Zimmermann} % oder \betreuerin{..}

%\strictlicense  % erzeugt restriktive Lizenzformel

%%%----------------------------------------------------------
\frontmatter
\maketitle
\setcounter{tocdepth}{1}
\tableofcontents
%%%----------------------------------------------------------

\chapter{Abstract}

Decomposing a software system into smaller parts has been an important discipline in our industry for many decades. With the rise of distributed systems, it has become more important to split a system into low coupled and high cohesive parts. Architectural styles like Service Oriented Architecture (SOA) and currently Microservices tackle the many challenges of such systems but remain vague on the art of how to decompose a system into services.

With the help of our industry partner Zühlke, our supervisor Prof. Dr. Olaf Zimmermann, and existing literature, we introduce a structured way to service decomposition by providing a comprehensive coupling criteria catalog.

We embodied these coupling criteria in the Service Cutter, a prototype that extracts coupling information out of well-known concepts like domain models and use cases. Using this information, the Service Cutter’s mission is to produce service cuts to assist an architect’s decomposition decisions. 

A scoring system is defined to automatically interpret coupling data. By employing a weighted, undirected graph and clustering algorithms the Service Cutter produces service cuts that minimize coupling between services while ensuring high cohesion within a service. 

Tests with two sample projects not only met our expectations but produced reasonable service cuts that have not been considered before. 

With a more structured way to service decomposition, the Service Cutter demonstrate that automated decision assistance is a promising way. The thesis lays the foundation for further projects in this area. 
			

%%%----------------------------------------------------------
\mainmatter         % Hauptteil (ab hier arab. Seitenzahlen)
%%%----------------------------------------------------------

% Don't show "Chapter X" titles for every Chapter
\makeatletter
\renewcommand{\@makechapterhead}[1]{%
\vspace*{50 pt}%
{\setlength{\parindent}{0pt} \raggedright \normalfont
\bfseries\Huge\thechapter.\ #1
\par\nobreak\vspace{40 pt}}}
\makeatother


\chapter{Introduction}

Starting with Microservices, what news they bring (deployability) and what old, underlying problem they try to solve (decomposition)

These: There is more to service decomposition than "Belongs to bigger concept defined by noun or verb(Chris Richardson)
=> Goal/Idea

- Identify coupling criteria and constraints after which service 
decomposition should be done. 

- Build a tool that takes data fields and applied coupling criteria and constraints as input and produces suggestions on how to split the system in different parts. 


\section{Scope}

Identification \& Implementation not in scope

Focus only on data, not on logic. (??)

No focus on ways to handle coupling (async, interfaces, messaging, CAP, REST, etc.)


We should define what is part of our scope and what not. We focus only on functional partitioning (Y-axis scalability) and neglect hardware or technological impacts on service decomposition. We’re also not going into how to implement or soften service dependencies with events, RPC, REST, Messaging, abstract interfaces etc. 

TODO: is "logic" in scope of the thesis? Sometimes a bounded context might depend on a lot of data fields of another BC, but in the ends all it want's is a single processed result of this number that could be served by the other BC. How do we handle this? => Could this be modelled as a data field itself with a new kind of coupling criteria "calculated by"?

Begriffe: Module, Service, Microservice, Komponente, DDD, Bounded Context, Entity, Aggregate, Classes, Packages
=> Microservice vs. Bounded Context vs. Service definition by Udi Dahan, what is it that we produce?
=> Bounded context is not really accurate. it defines the logical borders of a system, but there might be other coupling criteria (not logical ones?) that define the broders of a "thing"

Service
Over-defined but Udi Dahan uses a good definition that suits our result. 
=> Fowler Definition 
Service Boundary Recommondations
Service/BC > Microservices 
AppArch 3 Ebenen von Services
Drei definitionen von SOMA/Services, User / Architect / Developer


Entity: Lifecycle management \& identify (=> composition?)
Local: Module, remote: Microservice


\chapter{Analysis}
\label{cha:analysis}


\chapter{Requirements}
\label{cha:requirements}

These chapters describes the (non-)functional requirements of the target solutions and the target users working with the end product.

The requirements in this chapter are not prioritised. As part of the sprint planning meetings, these requirements are transformed and prioritised to user stories and tasks. The description in this chapter is meant to provide a high level overview and establish a common sense between all stakeholders of this project.



\section{Personas}

The personas are inspired by a series of discussions in meetings and workshops with our stakeholders.

All users of the system are experienced software architects or developers interested in the educational aspects of the tool.

\subsection{Junior Jedi-Master}

Junior has been a fast learning and aspiring developer since he graduated from university with a master's degree in information technology eight years ago. In his new job, Junior finds himself in the role of an architect for a new and promising product that enjoys the support of well known investors. In consequence of his experience in distributed systems, he has been assigned with the task to decompose the system's business model into logical services of which each will be split into multiple separately deployable microservices. 

Junior strongly believes in automation and using every tool available to support and complete his work. For his current project he plans to try the Service Cutter as a foundation and verification of his architectural decisions. 


\subsection{Walter Wisenheimer}

Walter is an architect with many years of experience in the industry and has built numerous systems already. Walter has seen many automation concepts and tools failing their goals. In Walter's view a natural and obvious outcome -- an architect's world is too complex to model and automate in a system or algorithm. 

After a longer discussion with an very motivated junior developer who recently joined his company, Walter starts to see the benefits of the well structured format the Service Cutter organizes architecturally relevant information. Using the tool to structure and document his systems characteristics might be of benefit as currently a lot of his precious knowledge remains tacit\cite{zimmermann2015architectural}. 

Walter decided to try the Service Cutter to structure the information for the current project he is working on.


\subsection{Stan Student}

Stan studies Computer Science and as part of his class in Service oriented Software Architecture he is supposed to design a set of services for the Cargo Tracking\cite{dddGithub} domain. The Service Cutter guides him through the important decisions by asking a set of questions and presents him a set of possible service cuts. 

Being overwhelmed of all the data requested by the Service Cutter he would like to configure the tool to only focus on the data he got provided in his exercises. 

Stan then discusses the advantages and disadvantages of the presented options with his fellow students. He furthermore asks his Professor about the to him unknown criteria the Service Cutter requested and why that information might have an impact on service orientation. 


\subsection{Tom Tutor}

Tom wants to introduce his students to the software architecture craft. He uses the Service Cutter to visualize the different ways of distribute data into services during his lectures. By changing the calculation parameters he can demonstrate that software architecture mostly depends on the context of the requirement. The same problem might have different solutions in varying circumstances.

TODO: add enterprise architect role here?
%TODO: enterprise architect role?

\section{Functional Requirements}

The Service Cutter faces the functional requirements presented in this
section.

%TODO low prio: Show warnings as part of the output should a predefined service have a significant negative effect!

\subsection{Coupling Criteria}

All \textit{Cohesiveness}, \textit{Compatibiltiy} and \textit{Constraints} criteria must be supported. The Service Cutter needs to rate for each criteria which nanoentities should be placed in one service and which need to be separated over multiple services. Within this rating, every coupling criterion is equally respected.

Criteria of type \textit{Communication} do not describe the need to merge or separate nanoentities but characterize nanoentities suitable for inter service communication and are handled with a lower priority than the other criteria. 

\subsection{User Representations}

To achieve better usability, the user does not need to now the exact definitions of the coupling criteria or their internal structure. He can use well known concepts called \textit{user representations} to describe his system, from which the relevant nanoentities and coupling criteria data will be extracted by the Service Cutter. At least the following User Representation must be supported by the system:

\begin{description}
	\item[Entity Relationship Diagram] containing \glspl{entity} and their relations to each other. Each entity contains a list of nanoentities building the basis for a systems analysis.
	\item[Use Case] containing at least information about all nanoentities read or written within a particular case. 
	\item[Categorization] of the nanoentities in different characteristics per compatibility coupling criteria. 
\end{description}

\subsection{Priorities}

As every coupling criteria is respected equally in the rating process, the user needs an additional way to prioritize criteria according to its system characteristics. For an application processing financial data, security might be more important than network traffic. For an application that needs to support high volumes of data, volatility and resilience might be a primary focus.

The system should allow to change priorities for all supported coupling criteria while providing reasonable default values. 

\subsection{Candidate Service Cuts}

Based on the input provided by user representations and the defined criteria priorities, the Service Cutter produces a set of candidate service cuts for the analyzed system. The requirements for such are documented in Section \ref{sec:decompositionRequirements}

\subsection{Visualize Published Language}

If relevant user input (e.g. use case definitions) is available, the Service Cutter is able to tell which service depends on which nanoentities of other services. These dependencies need to be visualized. All nanoentities shared between different services make up the published language of the system and need to be visualized as such.

\section{Nonfunctional Requirements}

The following non-functional requirements should be satisfied by the Service Cutter.

\subsection{Deployment}

No software installation is required. The latest version of any popular internet browser (Chrome, Firefox, Internet Explorer) is sufficient.

On the server side Java is used to implement the business logic. An appropriate storage and web server technology will be evaluated as part of the design phase. The deployment of the different components is managed using \gls{docker} containers.

\subsection{Usability}
\label{sec:usability}

A software architect should be able to use the software without any training. All controls are clearly named and, where appropriate, documented using an inline user manual.

Up to 500 data fields and 50 entities are manageable without losing track.

The base layout is responsive and adapts to smaller screens such as smartphones. However the tool is mostly used on devices such as laptops and the controls are therefore optimized for use on screens that are at least 15 inches wide and used with a mouse and a keyboard.

\subsection{Simplicity}

A simple workflow can be achieved with a few clicks. All steps are provided with useful defaults that can be changed.

\subsection{Information Security}

The web application is secured using an authentication and authorization implementation. Any other internal components such as the database or web services are hidden behind the servers firewall and therefore do not need any special security measures.

The uploaded data models are initially shared amongst all registered users.

\subsection{Separation of Concerns}

The software is structured in a way that separates the application into several concerns running in separated containers that are managed individually.

\subsection{Performance}

All regular user interactions should not take more than one second. The service boundary calculation may take up to 10 seconds depending on the data model size. The benchmark is the upper limit outlined in section \ref{sec:usability}.

\subsection{Monitoring}

The software is not business critical and no explicit monitoring is required.

\subsection{Availability and Fault Tolerance}

The software does not implement any high availability measures. A common error handling is built into the application and ensures that operations continue even in case of unexpected input or state.

\subsection{Maintainability}

The application should leverage existing open source frameworks or libraries whereever possible to ensure a minimal maintenance effort.

\subsection{Logs}

Log files should be written using SLF4J\cite{slf4j} provided by Spring Boot\cite{springboot}.

\subsection{License}

No GPL? to be discussed %TODO


\bigskip

After defining all (non-)functional requirements, the next Chapter outlines design and implementation steps which were taken to satisfy these requirements.


\chapter{Design and Implementation}
\label{cha:implementation}


\section{Design} 


\subsection{Technology}


\subsection{Infrastructure}



\bigskip
After covering the important design and implementation aspects, the next chapter assesses the built solution described in this chapter against the defined requirements.

\chapter{Conclusion}


%%%----------------------------------------------------------
%%%Anhang
\appendix

\chapter{Project Management}
\label{cha:projectmgmt}

This Appendix documents aspects of project management like methodology, roles, environment, quality management, and risk management. The project was time boxed and started on \formatdate{14}{9}{2015} and ended on \formatdate{18}{12}{2015} which implies 14 working weeks. 


%\section{Project Management Methodology}

We chose Scrum as the project management methodology for this bachelor thesis. Scrum specifies an iterative and incremental approach which encourages a high involvement of the customer. This characteristics suit the requirements of a bachelor thesis well,
as it provides only a short project definition at project start and, in our case, requires close collaboration with our advisor and industry partner.

%
\section{Project Roles}

Scrum defines three project roles — the product owner, the scrum master and the development
team. As this bachelor thesis is an academical project, the scrum roles could
not be mapped one to one. This Section introduces the involved persons and their roles
in the project.

\subsection{HSR Supervisor}

\begin{minipage}[t]{0.25\textwidth}
	\vspace{0pt}
	\includegraphics[width=0.8\textwidth]{olz.jpg}
\end{minipage}
\begin{minipage}[t]{0.8\textwidth}
	\vspace{10pt}
	Prof. Dr. Olaf Zimmermann is the \gls{HSR} supervisor of this thesis and incorporates both, the role of the product owner and scrum master. \textit{Product} in this context refers to the thesis itself as well as the functional product. He ensures that all requirements by HSR for a bachelor thesis are met and decides in last instance about the scope of the thesis. 
	As supervisor and coach of the development team, Mr. Zimmermann also performs part of the scrum master role as he coaches the development team.
\end{minipage}


\subsection{Industry Partner}

Our industry partner Zühlke Engineering, represented by W. Giersche, ensured the functional relevancy of this thesis. Mr. Giersche contributed valuable experience from many software engineering projects. He represented part of the product owner role as he played an important role in the functional prioritization to maximize the business value. Being an experienced software architect, he also consulted our research of existing literature and contributed to the coupling criteria catalog.


\subsection{Project Team}

Michael Gysel and Lukas Kölbener formed the development team. They worked as an interdisciplinary team in which both were responsible for each part of the project. Both being Certified ScrumMasters\textregistered\cite{scrummaster}, they incorporated part of the scrum master role as they helped maintaining the product backlog and organized everything for correct sprint operation.

\begin{minipage}[t]{0.25\textwidth}
	\vspace{0pt}
	\includegraphics[width=0.8\textwidth]{lukas.jpg}
\end{minipage}
\begin{minipage}[t]{0.8\textwidth}
	\vspace{20pt}
	Lukas Kölbener is an information technology student at \gls{HSR} in his 9\textsuperscript{th} semester. He works part time as Java developer for Super Computing Systems AG in Zurich, building ticket vending machines for the public transport industry.
	\newline
\end{minipage}

\begin{minipage}[t]{0.25\textwidth}
	\vspace{0pt}
	\includegraphics[width=0.8\textwidth]{michi.jpg}
\end{minipage}
\begin{minipage}[t]{0.8\textwidth}
	\vspace{20pt}
	Michael Gysel is an information technology student at \gls{HSR} in his 9\textsuperscript{th} semester. He works part time as Java Developer for FIS, a global provider for banking and payments technologies.
	\newline
\end{minipage}

%\section{Development Environment}

Figure \ref{fig:devenvironment} outlines all components of the development environment. The Service Cutter is developed with Java 8 using Eclipse Mars~\cite{eclipsemars}. \gls{HSR} has provided a \gls{VM} on which the \gls{docker} images are running. On the same server a \gls{CI} Jenkins server pulls for changes from the Git repositories on GitHub to build and test the different projects. The Docker images are built using a Maven plugin on Jenkins and then pushed into the local Docker repository. Table \ref{tab:vm} indicates the software installed by us on the server.


\begin{table}[H]
\begin{center}
\begin{tabular}
{|p{100pt} p{80pt}|}
\hline \textbf{Software} & \textbf{Version} \\ 
\hline Java & 1.8.0\_60 \\ 
\hline Jenkins & 1.629 \\ 
\hline Docker & 1.8.2 \\ 
\hline Docker Compose & 1.4.1 \\
\hline Apache Maven & 3.0.5 \\
\hline Node.js & 0.10.25 \\
\hline NPM & 1.3.10 \\
\hline 
\end{tabular} 
\caption{Installed software on sinv-56064.edu.hsr.ch (152.96.56.64).}
\label{tab:vm}
\end{center}
\end{table}

\begin{figure}[H]
	\centering{\includegraphics[scale=0.7]{diagrams/DevelopmentEnvironment.pdf}}
	\caption{Development Environment}
	\label{fig:devenvironment}
\end{figure}

%\input{projectmgmt/qualitymgmt}
%\section{Sprint Plan}
\label{sec:projplan}

The project is organized in 4 regular sprints with each two weeks length. 

\begin{itemize}
\item 1st Sprint: \formatdate{14}{09}{2015} - \formatdate{25}{09}{2015}
\subitem Refinement of thesis concept
\subitem Implementation of prototype
\item 2nd Sprint: \formatdate{12}{10}{2015} - \formatdate{23}{10}{2015}
\subitem Development of algorithm
\subitem Development of Coupling Criteria catalog
\item 3rd Sprint: \formatdate{09}{11}{2015} - \formatdate{20}{11}{2015}
\subitem Finish functional requirements
\subitem Test significantly large project
\item 4th Sprint: \formatdate{30}{11}{2015} - \formatdate{11}{12}{2015}
\subitem Refine scoring system, fine tune sample models
\item 5th Sprint: \formatdate{14}{12}{2015} - \formatdate{18}{12}{2015} (one week)
\end{itemize}

The 5th sprint will solely be used to finish the documentation.

%TODO müssen wir beschreiben, was wir zwischen den sprints machen?
%\include{projectmgmt/risk_management}

%\include{anhang_a}	% 
%\include{anhang_b} % 
%\include{anhang_c} % 
%\include{anhang_d} % 
%\include{anhang_e} % 

%\renewcommand{\glossarypreamble}{Descriptions of Glossary items have mainly been taken from Wikipedia.org.\newline\newline}

\printglossaries


%%%----------------------------------------------------------
\MakeBibliography
%%%----------------------------------------------------------


\end{document}
