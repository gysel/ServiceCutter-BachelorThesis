\chapter{Graph Clustering Evaluation}
\label{appendix:graphClustering}
\label{appendix:graphClusteringAlgs}
%TODO: test if refs are correct

This Appendix documents the requirements and evaluation of clustering algorithms. 

\section{Requirements}

The requirements listed in Table \ref{tab:requirementsAlgorithm} should be met by the algorithm and its implementation:

\begin{table}[H]
	\centering
	\caption{Algorithm Requirements}
	\label{tab:requirementsAlgorithm}
	\begin{tabular}{|p{100pt}|p{250pt}|p{50pt}|}
		\hline	
		Name & Description & Priority \\
		\hline
		Distinct Clusters & Every nanoentity is contained once and only once in a cluster. & High  \\
		\hline
		Minimal Coupling & The total weight of the edges connecting clusters should be minimal. & High \\
		\hline
		Implementation & A free implementation of the algorithm should be available either in Java or another language easily callable from the \gls{JVM}. & High  \\
		\hline
		Number of Clusters & The number of clusters should be defined by a parameter. & Medium \\
		\hline
		Performance & The algorithm should not take longer than 2 minutes on an average computer to cluster 2000 nodes.& Medium \\
		\hline
		Simplicity & It should be possible to understand the mechanism and parameters of the algorithm within a day assuming the mathematical background of an average \gls{HSR} student. & Low \\
		\hline
		Edge Cases & Cases in which it is unclear which cut is best should be visualized in form of a hint or multiple solution suggestions. & Low \\
		\hline
		License & ? & ? \\
		\hline
	\end{tabular}
\end{table}

\section{Algorithms Assessment}

The algorithms listed in Table \ref{tab:algorithmEvaluation} were found by consulting clustering algorithm comparisons published by the Computer Science Review\cite{schaeffer2007graph} and the Physical Review E\cite{lancichinetti2009community}. Further Research using Google's search engine and the community driven question and answer platform Stackoverflow\cite{stackoverflowGraphClustering} were used to find applicable implementations of the algorithms. 

\begin{table}[H]
	\centering
	\caption{Algorithm Evaluation}
	\label{tab:algorithmEvaluation}
	\begin{tabular}{|p{60pt}|p{140pt}|p{130pt}|p{70pt}|}
		\hline	
		\textbf{Name} & \textbf{Description} & \textbf{Implementation} & \textbf{Assessment} \\ 
		\hline
		MCL - Markov Cluster Algorithm\cite{markovCluster} & A clustering algorithm working on weighted undirected graphs. MCL is based on Random Walks with Markov Chains. & Implementations of MCL are available in R and in Java as a plugin of the Gephi\cite{gephi} platform. & Positive \\
		\hline
		HCS - Highly Connected Subgraphs\cite{hcs} & A clustering algorithm working on unweighted undirected graphs. The CLICK clustering algorithm enhances HCS for weighted edges. & Implementations only available in R. & No Java implementation  \\
		\hline
		Girvan–\newline Newman\cite{girvan} & A clustering algorithm working on weighted undirected graphs based on Edge-Betweenness \footnote{The amount of shortest paths between nodes going through a specific edge.}. & Java implementations exist as part of the Jung\cite{jung} framework (only unweighted graphs) and as a plugin\cite{girvanGephi} of the Gephi\cite{gephi} platform. & Positive \\
		\hline	
		K-means\cite{kmeans} & A clustering algorithm working with vectors in an n-dimensional space. & Multiple implementations, for example as part of the Spark\cite{spark} framework, are available. & No simple way to transform the problem from a graph to vector based representation found. \\
		\hline
		Apiacoa\cite{apiacoa} & A clustering algorithm working on unweighted undirected graphs. This algorithm is based on maximal modularity clustering. & Apiacoa.org provides an implementation of the algorithm in Java. & No support for weighted edges\\
		\hline
		Epidemic Label Propagation & A clustering algorithm working on weighted (un-)directed graphs. This algorithm was suggested by Raghavan\cite{raghavan} and refined by Leung\cite{leung}. & GraphStream\cite{leungGraphstream} provides an implementation of the algorithm in Java as part of their \texttt{gs-algo} package. & Positive \\
		\hline
	\end{tabular}
\end{table}



