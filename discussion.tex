\chapter{Discussion}

This chapter discusses the outcome of the thesis, the Service Cutter and its impact for future work.


%TODO Broader algorithm testing. Also non-Java implementations

\section{Thesis Evaluation}

At the beginning of this project we formulated two hypothesizes. We were able to produce results proving both of them.
%TODO convert goals to reference?

%Die Gründe für Service Cuts in einem Softwaresystem kann man als Liste von \enquote{Coupling Criteria} umfassend beschreiben.
\begin{quote}
	\textit{The driving factors for service cuts in a software system can be compiled into a comprehensive criteria catalog.}
\end{quote}

We successfully compiled a catalog of 16 coupling criteria that aims to form a comprehensive but not conclusive collection. 

The catalog helps a software architect to structurize driving factors for service decomposition while providing a good system documentation. The developed criteria may provide a basis for a common language amongst architects. 

%Ziel erreicht. umfassend - aber nicht abschliessend. der katalog kann aus unserer sicht einen diskussionsbeitrag liefern und zum beispiel eine grundlage für eine gemeinsame sprache für architekten liefern.

\begin{quote}
	\textit{The data of the criteria catalog can be embodied in a tool to optimize loose coupling between and high cohesion within services in a structured and automated way.}
\end{quote}
%Die Informationen aus den \enquote{Coupling Criteria} können strukturiert verarbeitet werden, um die Entkopplung von Softwarekomponenten zu optimieren.}

In the Service Cutter assessment documented in Appendix \ref{appendix:serviceCutterAssessment}, we tested two sample systems each with the algorithms Girvan-Newman and Leung. Girvan-Newman provided expected and therefore satisfying results in only one of the two example systems. Leung on the other hand did not only provide expected service cuts for both systems but surprised us with suggestions that were unexpected but definitely reasonable.  

\section{Requirements Assessment}

This section assesses the developed solution based on the defined requirements. The two domain models as described in Appendix \ref{appendix:serviceCutterAssessment} as well as the implemented Service Cutter itself serve as the test scenario.

All requirements are rated with a rating from $1-3$.

\begin{description}
\item[1] The requirement is fully satisfied.
\item[2] The requirement is partially satisfied.
\item[3] The requirement is not satisfied.
\end{description}

\subsection{Functional Requirements}

Table \ref{tab:conclusionFunctional} assesses the provided solution against the defined functional requirements described in Section \ref{sec:functionalRequirements}.

\begin{table}[H]
	\centering
	\caption{Assessment of functional requirements}
	\label{tab:conclusionFunctional}
	\begin{tabular}{|p{100pt}|l|p{250pt}|}
	\hline \textbf{Requirement} & \textbf{Rating} & \textbf{Assessment} \\ 
	\hline Coupling Criteria & 1 & All coupling criteria have been implemented in the Service Cutter.  \\ %TODO stimmt das wirklich? noch nicht!
	\hline User Representations & 1 & All required user representations are supported by the importer of the Service Cutter. \\ 
	\hline Priorities & 1 & Priorities are built into the scoring process. \\ 
	\hline Candidate Service Cuts & 2 & The Service Cutter visualizes a candidate service cut using a chart. Multiple candidate cuts can only be compared using several browser tabs. \\ 
	\hline Published Language & 1 & The published language is extracted from the candidate service cut when selecting a service node in the visualization.  \\ 
	\hline Hard Architectural Decisions & 2 & This feature is not explicitly implemented but partially given by non-deterministic algorithms as discussed in Section \ref{subsec:algoDiscussion}\\
	\hline 
	\end{tabular} 
\end{table}

\clearpage
\subsection{Non-Functional Requirements}

Table \ref{tab:conclusionNonFunctional} assesses the provided solution against the defined non-functional requirements described in Chapter \ref{cha:requirements}, Section \ref{sec:nonfunctionalRequirements}. %TODO überall so? chapter X, section Y

\begin{table}[H]
	\centering
	\caption{Assessment of non-functional requirements}
	\label{tab:conclusionNonFunctional}
	\begin{tabular}{|p{100pt}|l|p{250pt}|}
	\hline \textbf{Requirement} & \textbf{Rating} & \textbf{Assessment} \\ 
	\hline Usability & 2 & We reviewed the user interface within the project team but did not conduct usability tests. \\
	\hline Simplicity & 1 & A simple analysis can be performed with 5 mouse clicks. \\
	\hline Performance & 2 & The sample domain models can be decomposed in not more than two seconds. Extensive performance tests have not been conducted due to time budget constraints as decided with our stakeholders. \\
	\hline Logging, \newline Deployment & 1 & Logging is based on SLF4J and a deployment based on Docker is implemented. \\
	\hline Fault Tolerance & 1 & All errors are handled and occuring errors are logged. \\
	\hline Maintainability & 1 & The implementation is based on two services (editor and engine) both built with suitable application layers. Communication between the layers is implemented with RESTful HTTP communication. Open source tools are used wherever possible. \\
	\hline State-of-the-Art Technology & 1 & The application is based on the technology stack suggested by our industry partner. \\
	\hline License & 1 & The source code has been released under the Apache 2.0 license. \\
	\hline 
	\end{tabular} 
\end{table}


\section{Conclusion}

The Service Cutter as of now does not yet provide significant business value to its users. It is not a production grade architectural tool but a proof that our concept of structured and automated decomposition optimization generally works and is worth further investigations. We believe that the great value to its users will become apparent when further efforts are put in the following four aspects:

\begin{enumerate}
	\item The \textbf{scoring process} for the different type of criteria should further be analyzed and tested. More sophisticated solutions for conceptual challenges like the single dimensionality problem documented in Section \ref{subsec:singleDimensionality} could improve the accuracy and meaningfulness of results. 
	\item \textbf{Graph clustering algorithms} should further be analyzed and optimized. Possible alternative approaches as documented in Appendix \ref{appendix:decompositionAppraoches} could furthermore solve some of the conceptual challenges of the scoring process.
	\item The \textbf{Service Cutter Prototype} should be enhanced to a production ready tool with graphical user interfaces for defining, editing, and storing a user's system description and candidate service cuts. %TODO storing user representations?
	\item The idea of the Service Cutter should be integrated into a \textbf{toolkit chain}. The input could automatically be generated from other tools or diagrams and the output used for code or \gls{API} generation.
\end{enumerate}

\bigskip
This chapter assessed our catalog of coupling criteria and the Service Cutter. The next chapter proposes a series of possible enhancements in the area of structured service decomposition.
